\section{Hastighetsregulator}\label{sec:hastighetsreg}

\subsection{Tabeller}
Under finner dere kildekoden til Tabell \ref{tab:eksempeltabell}


\begin{table}[tb]
	\centering
	\begin{tabular}{ll} 
	    % ll står for to venstrejusterte kolonner. For høyrejustering kan dere bruke r i stedet. Disse kan selvsagt blandes etter behov. Vi skiller kolonner med &, og avslutter linjer med \\, som dere ser under.
		\toprule
		Variabel & Verdi \\
		\midrule
        $K_p$ & $10.0$ \\
        $K_i$ & $2.0$\\
		\bottomrule
	\end{tabular}
    \caption{Et eksempel på en tabell med regulatorparametre. Merk både at tabellteksten plasseres over tabellen, og at vi vanligvis ikke bruker loddrette linjer.}
\label{tab:eksempeltabell}
\end{table}

\subsection{Sitering og referansehåndtering}
I akademisk skriving er det veldig viktig å sitere kilder. I latex har vi et verktøy kalt BiBTeX som 
håndterer dette. Kilden legges til i en .bib-fil, i vårt tilfelle \texttt{bibliography.bib}. Der kan dere
se hvordan kilder legges inn. For å sitere kilden kan du bruke kommandoen \texttt{\textbackslash{cite}}.
Merk at kilder som ikke siteres, ikke vil dukke opp i kildelisten! Dette betyr at dere i bibliografi-filen kan
legge inn alle kilder dere \emph{tror} dere kommer til å bruke, men de vil ikke dukke opp i kildelisten før du
faktisk siterer dem!

Det finnes mange forskjellige siteringsstiler, og dere kan modifisere mye. Hvis dere vil vite mer om dette
kan dere finne informasjon her \cite{BiberBibtexEtc,WikibookLatex}\footnote{Prøv å begrense sitering av 
nettsider så mye som mulig, og bruk eventuelt \url{http://web.archive.org} hvis dere er bekymret for at en 
referanse kan endre seg i fremtiden.}.

Det finnes mange verktøy for å håndtere referanser. 
De fleste av dere vil nok ikke trenge dem i dette faget, men vi legger ved informasjon, så dere vet om dem 
senere. Noen populære referanseverktøy er JabRef (\url{http://www.jabref.org/}), Mendeley (\url{https://www.mendeley.com/}), EndNote, og Zotero. Alle disse kan eksportere .bib-filer (JabRef lagrer alle
data i en .bib-fil).
