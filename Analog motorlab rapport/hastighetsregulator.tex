\section{Hastighetsregulator}\label{sec:hastighetsreg}


\subsection{Teori}

Regulatorer brukes for å styre tilstander i en prosess. En regulator regulator får inn et avvik mellom referansen og den målte tilstanden. Regulatorer bruker avviket til å beregne et pådrag til prosessen. P-regulator er en type regulator som lager et pådrag, $u$,  som er proposjonalt med avviket, $e = \omega_d - \omega_m$. $K_p$ er proposjonalitetsleddet i regulatoren, sammenhengen er vist i likning \eqref{eq:P_regulator}.

\begin{equation}
    \label{eq:P_regulator}
    u(t) = K_p e(t)
\end{equation}

En slik regulator er effektiv så lenge det ikke er en kraft som forhindrer systemet å oppnå en spesifikk tilstand. Da vil ikke regulatoren regulere tilstanden helt til referanseverdien og det oppstår et stasjonærtavvik. Friksjon og luftmotstand er krefter som vil motvirke at et system oppnår en hasighet større enn $0$. Et integratorledd vil motvirke slike krefter, ved å gi et pådrag lik som disse kreftene. Da vil tilstanden nå referansetilstanden og det stasjonæreavviket forsvinner. Likning \eqref{eq:PI_regulator} beskriver en PI-regulator,

\begin{equation}
    \label{eq:PI_regulator}
    u(t) = K_p e(t) + \frac{K_p}{T_i} \int_{0}^{t} e(\tau) d\tau,
\end{equation}

der $u$ er pådraget, $K_p$ er proposjonalitetskontanten, $e$ er avviket mellom referansen og tilstanden og $T_i$ er integratorkonstanten som representerer tidskonstanten til regulatoren.

\subsection{Metode}

\begin{figure}[b]
    \centering
    \begin{circuitikz} [scale=0.5, transform shape]
    \ctikzset{resistor = european}

    % --- OP1 ---
    \node[op amp](OP1) {$OP1$};
    
    \draw (OP1.-)
    to[R, l_=$R_1$] ++(-2, 0)
    to[short, o-, l=$\omega_m$] ++(0, 0);

    \draw (OP1.+)
    to[R=$R_1$] ++(-2, 0)
    to[short, o-, l=$\omega_d$] ++(0, 0);

    \draw (OP1.+)
    to[R=$R_2$, *-] ++(0, -2)
    node[ground] {};

    \draw (OP1.-)
    to[short, *-] ++(0, 1)
    coordinate(t1)
    to[R=$R_2$] (t1 -| OP1.out)
    -- (OP1.out);

    % --- OP2 ---

    \draw (OP1.out)
    to[short, *-, l=$e$] ++(0.5, 0)
    to[R=$R_3$] ++(2, 0)
    node[op amp, anchor=-](OP2) {$OP2$};

    \draw (OP2.-)
    to[short, *-] ++(0, 1)
    coordinate(t2)
    to[R=$R_4$] (t2 -| OP2.out)
    -- (OP2.out);

    \draw (OP2.+)
    node[ground] {};

    \draw (OP2.out)
    to[short, *-] ++(0, 0)
    to[R=$R_7$] ++(0, -2)
    coordinate(t7);

    % --- OP3 ---

    \draw (OP1.out)
    -- ++(0, -2)
    to[potentiometer, n=R5, l=$R_5$] ++(0, -2)
    coordinate(t3);

    \draw (R5.wiper)
    -- (R5.wiper |- t3)
    -- (t3);

    \draw (t3)
    to[short, *-] ++(0, -1)
    to[R=$R_6$] ++(2, 0)
    node[op amp, anchor=-](OP3) {$OP3$};

    \draw (OP3.+)
    node[ground] {};
    
    \draw (OP3.-)
    to[short, *-] ++(0, 1)
    coordinate(t4)
    to[C=$C_1$] (t4 -| OP3.out)
    coordinate(t5)
    -- (OP3.out);

    \draw (t4)
    to[short, *-] ++(0, 1.3)
    coordinate(t6)
    to[open jumper, l=$JP1$] (t6 -| t5)
    to[short, -*] (t5);

    \draw (OP3.out)
    to[short, *-] (OP3.out -| OP2.out)
    to[R, l_=$R_7$] ++(0, 2)
    coordinate (t8)
    to[short, *-*] (t7);

    % --- OP4 ---

    \draw (t8)
    -- ++(1.5, 0)
    node[op amp, anchor=-](OP4) {$OP4$};

    \draw (OP4.+)
    node[ground] {};

    \draw (t7)
    to[R=$R_8$] ++(2, 0)
    coordinate(t9)
    to[potentiometer, n=R9, l=$R_9$] (t9 -| OP4.out)
    coordinate(t10)
    -- (OP4.out);

    \draw (R9.wiper)
    -- (R9.wiper -| t10)
    to[short, -*] (t10);

    \draw (OP4.out)
    to[short, *-o, l=$V_m$] ++(1, 0);
    
\end{circuitikz}
    \caption{PI-regulator krets for hastighetsregulatoren. Hentet fra \cite{AnalogMotorlabbOppgaver}}
    \label{fig:krets_hastighets_regulator}
\end{figure}

Hastighetsregulatoren ble implementert som en analog PI-regulator som vist i \autoref{fig:krets_hastighets_regulator}. $OP1$ er en differensialforsterker som finner avviket, $e$, transferfunksjonen er gitt ved \autoref{eq:differensialforsterker}.
$OP2$ er en inverterende forsterker som inverterer avviket, uten forsterkning eller demping.
$OP3$ er en integrerende forsterker som integrerer $e$ og forsterker den med $\frac{1}{T_i}$. $JP1$ brukes for å nullstille integratoren og skru av I-leddet i integratoren. Da vil regulatoren oppføre seg som en P-regulator. Transferfunksjonen til $OP3$ er $-\frac{1}{(R_5 + R_6) C_1} \int e dt$. Dersom $JP1$ er kortsluttet er speningen ut fra $OP3$ alltid $0\,V$
$OP4$ summerer spenningen fra $OP2$ og $OP3$ og forsterker resultatet med $K_p$. Transferfunksjonen for $OP4$ er $-\frac{R_8 + R_9}{R_7}(v_2 + v_3)$, der $v_2$ er spenningen ut av $OP2$ og $v_3$ er spenningen ut av $OP3$. Ut fra dette finner vi uttrykk for $K_p$ og $T_i$ som vist i \autoref{eq:K_p_og_T_i}.

\begin{equation}
    \label{eq:K_p_og_T_i}
    K_p = \frac{R_2}{R_1} \frac{R_8 + R_9}{R_7},
    T_i = (R_5 + R_6) C_1
\end{equation}

Størrelsen på motstandene og kondensatoren er vist i \autoref{tab:Komponenter_i_hastighetsregulatoren}

\begin{table}
    \centering
    \caption{Motstander og kondensatorer i hastighetsregulatoren. Verdiene er hentet fra \cite{AnalogMotorlabbOppgaver}}
    \begin{tabular}{lll}
        \toprule
        Størrelse & Verdi & Type \\
		\midrule
        $R_1$, $R_2$ & \SI{100}{\kilo\ohm} & Resistor\\
        $R_3$, $R_4$, $R_7$, $R_8$ & \SI{10}{\kilo\ohm} & Resistor \\
        $R_5$, $R_9$ & \SI{1}{\mega\ohm} & Potmeter \\
        $R_6$ & \SI{1}{\kilo\ohm} & Resistor \\
        $C_1$ & \SI{1}{\micro\farad} & Kondensator \\
        \bottomrule
    \end{tabular}
    \label{tab:Komponenter_i_hastighetsregulatoren}
\end{table}

\subsection{Resultater}
\todo[inline]{Vi språk i resultater? eller typ 'det ble observert'? }
\label{obs:hastighet_regulator_breming_med_finger}
Vi observerte at stasjonæravviket og pådraget økte, dersom vi bremset motoren med en finger ved bruk av P-regulator. Ved bruk av PI-regulator økte pådraget til tilstanden nådde referanseverdien, selv med en finger som bremset motoren.

\begin{figure}[h]
    \centering
    % This file was created with tikzplotlib v0.10.1.
% Dette er eksempel data
\begin{tikzpicture}

\definecolor{darkgray176}{RGB}{176,176,176}
\definecolor{darkorange25512714}{RGB}{255,127,14}
\definecolor{steelblue31119180}{RGB}{31,119,180}

\begin{axis}[
tick align=outside,
tick pos=left,
title={Hastighet P-regulator},
legend pos=south east,
height=\figH,
width=\figW,
x grid style={darkgray176},
xlabel={Tid [ms]},
xmin=-11.54538, xmax=242.45298,
xtick style={color=black},
xtick={-50,0,50,100,150,200,250},
xticklabels={
  \(\displaystyle {\ensuremath{-}50}\),
  \(\displaystyle {0}\),
  \(\displaystyle {50}\),
  \(\displaystyle {100}\),
  \(\displaystyle {150}\),
  \(\displaystyle {200}\),
  \(\displaystyle {250}\)
},
y grid style={darkgray176},
ylabel={Spenning [V]},
ymin=-10.231936, ymax=10.231936,
ytick style={color=black},
ytick={-12.5,-10,-7.5,-5,-2.5,0,2.5,5,7.5,10,12.5},
yticklabels={
  \(\displaystyle {\ensuremath{-}12.5}\),
  \(\displaystyle {\ensuremath{-}10.0}\),
  \(\displaystyle {\ensuremath{-}7.5}\),
  \(\displaystyle {\ensuremath{-}5.0}\),
  \(\displaystyle {\ensuremath{-}2.5}\),
  \(\displaystyle {0.0}\),
  \(\displaystyle {2.5}\),
  \(\displaystyle {5.0}\),
  \(\displaystyle {7.5}\),
  \(\displaystyle {10.0}\),
  \(\displaystyle {12.5}\)
}
]
\legend{$\omega_m$, $\omega_d$}
\addplot [semithick, steelblue31119180]
table {%
0 -8.42285
0.2772000000002 -8.37402
0.554400000000399 -8.22754
0.83159999999971 -8.47168
1.10879999999991 -8.42285
1.38600000000011 -8.42285
1.66320000000031 -8.37402
1.94039999999962 -8.12988
2.21759999999982 -8.22754
2.49480000000002 -8.47168
2.77200000000022 -8.37402
3.04919999999953 -8.47168
3.32639999999973 -8.22754
3.60359999999993 -8.42285
3.88080000000013 -8.3252
4.15800000000033 -8.61816
4.43519999999964 -8.3252
4.71239999999984 -8.27637
4.98960000000004 -8.37402
5.26680000000024 -8.47168
5.54399999999955 -8.52051
5.82119999999975 -8.42285
6.09839999999995 -8.3252
6.37560000000015 -8.12988
6.65280000000035 -8.37402
6.92999999999966 -8.42285
7.20719999999986 -8.37402
7.48440000000006 -8.3252
7.76160000000026 -8.17871
8.03879999999957 -8.37402
8.31599999999977 -8.37402
8.59319999999997 -8.42285
8.87040000000017 -8.42285
9.14760000000037 -8.42285
9.42479999999968 -8.52051
9.70199999999988 -8.47168
9.97920000000008 -8.22754
10.2564000000003 -8.3252
10.5335999999996 -8.37402
10.8107999999998 -8.42285
11.088 -8.47168
11.3652000000002 -8.61816
11.6424000000004 -8.22754
11.9195999999997 -8.17871
12.1967999999999 -8.27637
12.4740000000001 -8.42285
12.7512000000003 -8.3252
13.0283999999996 -8.42285
13.3055999999998 -8.27637
13.5828 -8.42285
13.8600000000002 -8.37402
14.1371999999995 -8.61816
14.4143999999997 -8.3252
14.6915999999999 -8.22754
14.9688000000001 -8.42285
15.2460000000003 -8.47168
15.5231999999996 -8.42285
15.8003999999998 -8.47168
16.0776 -8.22754
16.3548000000002 -8.42285
16.6319999999995 -8.56934
16.9091999999997 -8.3252
17.1863999999999 -8.37402
17.4636000000001 -8.3252
17.7408000000003 -8.27637
18.0179999999996 -8.47168
18.2951999999998 -8.3252
18.5724 -8.47168
18.8496000000002 -8.17871
19.1267999999996 -8.42285
19.4039999999998 -8.37402
19.6812 -8.42285
19.9584000000002 -8.22754
20.2356000000004 -8.12988
20.5127999999997 -8.42285
20.7899999999999 -8.61816
21.0672000000001 -8.52051
21.3444000000003 -8.47168
21.6215999999996 -8.08105
21.8987999999998 -8.12988
22.176 -8.3252
22.4532000000002 -8.22754
22.7304000000004 -8.3252
23.0075999999997 -8.37402
23.2847999999999 -8.17871
23.5620000000001 -8.3252
23.8392000000003 -8.22754
24.1163999999996 -8.42285
24.3935999999998 -8.27637
24.6708 -8.17871
24.9480000000002 -8.3252
25.2252000000004 -8.61816
25.5023999999997 -8.27637
25.7795999999999 -8.42285
26.0568000000001 -8.27637
26.3340000000003 -8.47168
26.6111999999996 -8.52051
26.8883999999998 -8.37402
27.1656 -8.22754
27.4428000000002 -8.42285
27.7199999999995 -8.42285
27.9971999999997 -8.42285
28.2743999999999 -8.37402
28.5516000000001 -8.42285
28.8288000000003 -8.42285
29.1059999999996 -8.27637
29.3831999999998 -8.52051
29.6604 -8.27637
29.9376000000002 -8.3252
30.2147999999995 -8.42285
30.4919999999997 -8.42285
30.7691999999999 -8.42285
31.0464000000001 -8.3252
31.3236000000003 -8.56934
31.6007999999997 -8.03223
31.8779999999999 -8.42285
32.1552000000001 -8.22754
32.4324000000002 -8.22754
32.7095999999996 -8.08105
32.9867999999998 -8.42285
33.264 -8.42285
33.5412000000002 -8.42285
33.8184000000004 -8.47168
34.0955999999997 -8.52051
34.3727999999999 -8.3252
34.6500000000001 -8.47168
34.9272000000003 -8.3252
35.2043999999996 -8.56934
35.4815999999998 -8.37402
35.7588 -8.37402
36.0360000000002 -8.3252
36.3132000000004 -8.61816
36.5903999999997 -8.3252
36.8675999999999 -8.27637
37.1448000000001 -8.22754
37.4220000000003 -8.42285
37.6991999999996 -8.56934
37.9763999999998 -8.56934
38.2536 -8.42285
38.5308000000002 -8.12988
38.8080000000004 -8.37402
39.0851999999997 -8.42285
39.3623999999999 -8.27637
39.6396000000001 -8.3252
39.9168000000003 -8.22754
40.1939999999996 -8.47168
40.4711999999998 -8.42285
40.7484 -8.42285
41.0256000000002 -8.22754
41.3027999999995 -7.93457
41.5799999999997 -8.37402
41.8571999999999 -8.42285
42.1344000000001 -8.52051
42.4116000000003 -8.22754
42.6887999999996 -8.22754
42.9659999999998 -8.42285
43.2432 -8.52051
43.5204000000002 -8.37402
43.7975999999995 -8.37402
44.0747999999997 -8.27637
44.3519999999999 -8.3252
44.6292000000001 -8.61816
44.9064000000003 -8.52051
45.1835999999997 -8.66699
45.4607999999999 -8.3252
45.7380000000001 -8.22754
46.0152000000003 -8.37402
46.2923999999996 -8.3252
46.5695999999998 -8.42285
46.8468 -8.56934
47.1240000000002 -8.52051
47.4012000000004 -8.42285
47.6783999999997 -8.3252
47.9555999999999 -8.42285
48.2328000000001 -8.08105
48.5100000000003 -8.3252
48.7871999999996 -8.37402
49.0643999999998 -8.42285
49.3416 -8.37402
49.6188000000002 -8.3252
49.8960000000004 -8.42285
50.1731999999997 -8.81348
50.4503999999999 -8.52051
50.7276000000001 -8.22754
51.0048000000003 -8.27637
51.2819999999996 -8.22754
51.5591999999998 -8.22754
51.8364 -8.42285
52.1136000000002 -8.3252
52.3907999999995 -8.17871
52.6679999999997 -8.22754
52.9451999999999 -8.42285
53.2224000000001 -8.47168
53.4996000000003 -8.27637
53.7767999999996 -8.27637
54.0539999999998 -8.27637
54.3312 -8.37402
54.6084000000002 -8.61816
54.8855999999995 -8.27637
55.1627999999997 -8.37402
55.4399999999999 -8.08105
55.7172000000001 -8.37402
55.9944000000003 -8.3252
56.2715999999996 -8.47168
56.5487999999998 -8.3252
56.826 -8.22754
57.1032000000002 -8.37402
57.3803999999996 -8.61816
57.6575999999998 -8.37402
57.9348 -8.47168
58.2120000000002 -8.27637
58.4892000000004 -8.52051
58.7663999999997 -8.37402
59.0435999999999 -8.42285
59.3208000000001 -8.27637
59.5980000000003 -8.12988
59.8751999999996 -8.42285
60.1523999999998 -8.3252
60.4296 -8.42285
60.7068000000002 -8.42285
60.9840000000004 -8.17871
61.2611999999997 -8.27637
61.5383999999999 -8.42285
61.8156000000001 -8.42285
62.0928000000003 -8.22754
62.3699999999996 -8.22754
62.6471999999998 -8.3252
62.9244 -8.42285
63.2016000000002 -8.27637
63.4788000000004 -8.42285
63.7559999999997 -8.3252
64.0331999999999 -8.37402
64.3104000000001 -8.42285
64.5876000000003 -8.42285
64.8647999999996 -8.27637
65.1419999999998 -8.42285
65.4192 -8.42285
65.6964000000002 -8.47168
65.9735999999995 -8.3252
66.2507999999997 -8.42285
66.5279999999999 -8.22754
66.8052000000001 -8.42285
67.0824000000003 -8.42285
67.3595999999996 -8.56934
67.6367999999998 -8.37402
67.914 -8.3252
68.1912000000002 -8.42285
68.4683999999995 -8.42285
68.7455999999997 -8.37402
69.0227999999999 -8.47168
69.3000000000001 -8.27637
69.5772000000003 -8.47168
69.8543999999997 -8.42285
70.1315999999998 -8.37402
70.4088 -8.47168
70.6860000000002 -8.22754
70.9631999999996 -8.22754
71.2403999999998 -8.52051
71.5176 -8.3252
71.7948000000002 -8.17871
72.0720000000004 -8.12988
72.3491999999997 -8.42285
72.6263999999999 -8.27637
72.9036000000001 -8.52051
73.1808000000003 -8.37402
73.4579999999996 -8.42285
73.7351999999998 -8.37402
74.0124 -8.47168
74.2896000000002 -8.22754
74.5668000000004 -8.47168
74.8439999999997 -8.12988
75.1211999999999 -8.42285
75.3984000000001 -8.42285
75.6756000000003 -8.52051
75.9527999999996 -8.37402
76.2299999999998 -8.37402
76.5072 -8.22754
76.7844000000002 -8.56934
77.0616000000004 -8.47168
77.3387999999997 -8.47168
77.6159999999999 -8.08105
77.8932000000001 -8.22754
78.1704000000003 -8.52051
78.4475999999996 -8.61816
78.7247999999998 -8.37402
79.002 -8.22754
79.2792000000002 -8.08105
79.5563999999995 -8.03223
79.8335999999997 -7.93457
80.1107999999999 -7.88574
80.3880000000001 -7.59277
80.6652000000003 -7.34863
80.9423999999996 -7.34863
81.2195999999998 -7.20215
81.4968 -7.15332
81.7740000000002 -7.25098
82.0511999999995 -6.90918
82.3283999999997 -6.81152
82.6056 -6.81152
82.8828000000001 -6.66504
83.1600000000003 -6.7627
83.4371999999996 -6.46973
83.7143999999999 -6.4209
83.9916000000001 -6.0791
84.2688000000003 -6.22559
84.5459999999996 -6.12793
84.8231999999998 -6.03027
85.1004 -5.88379
85.3776000000002 -5.78613
85.6548000000004 -5.83496
85.9319999999997 -5.49316
86.2091999999999 -5.34668
86.4864000000001 -5.29785
86.7636000000003 -5.2002
87.0407999999996 -5.24902
87.3179999999998 -5.00488
87.5952 -5.00488
87.8724000000002 -4.8584
88.1496000000004 -4.8584
88.4267999999997 -4.76074
88.7039999999999 -4.71191
88.9812000000001 -4.41895
89.2584000000003 -4.37012
89.5355999999996 -4.27246
89.8127999999998 -4.22363
90.09 -4.12598
90.3672000000002 -4.22363
90.6443999999995 -3.88184
90.9215999999997 -3.78418
91.1987999999999 -3.6377
91.4760000000001 -3.73535
91.7532000000003 -3.49121
92.0303999999996 -3.34473
92.3075999999998 -3.34473
92.5848 -3.19824
92.8620000000002 -3.00293
93.1391999999995 -2.90527
93.4163999999997 -2.75879
93.6935999999999 -2.80762
93.9708000000001 -2.56348
94.2480000000003 -2.51465
94.5251999999996 -2.36816
94.8023999999998 -2.17285
95.0796 -2.22168
95.3568000000002 -2.17285
95.6339999999996 -1.97754
95.9111999999998 -1.92871
96.1883999999999 -1.78223
96.4656000000002 -1.78223
96.7428000000004 -1.53809
97.0199999999997 -1.48926
97.2971999999999 -1.48926
97.5744000000001 -1.34277
97.8516000000003 -1.29395
98.1287999999996 -1.24512
98.4059999999998 -1.00098
98.6832 -0.952148
98.9604000000002 -0.952148
99.2376000000004 -0.708008
99.5147999999997 -0.65918
99.7919999999999 -0.512695
100.0692 -0.561523
100.3464 -0.268555
100.6236 -0.268555
100.9008 -0.12207
101.178 -0.0732422
101.4552 0.12207
101.7324 0.170898
102.0096 0.415039
102.2868 0.366211
102.564 0.366211
102.8412 0.561523
103.1184 0.708008
103.3956 0.805664
103.6728 0.805664
103.95 0.854492
104.2272 1.19629
104.5044 1.29395
104.7816 1.29395
105.0588 1.44043
105.336 1.44043
105.6132 1.44043
105.8904 1.58691
106.1676 1.7334
106.4448 1.83105
106.722 1.83105
106.9992 1.97754
107.2764 1.97754
107.5536 1.97754
107.8308 2.31934
108.108 2.41699
108.3852 2.36816
108.6624 2.51465
108.9396 2.6123
109.2168 2.66113
109.494 2.85645
109.7712 2.85645
110.0484 2.85645
110.3256 3.10059
110.6028 2.9541
110.88 3.14941
111.1572 3.19824
111.4344 3.39355
111.7116 3.44238
111.9888 3.54004
112.266 3.6377
112.5432 3.6377
112.8204 3.93066
113.0976 3.83301
113.3748 4.02832
113.652 4.12598
113.9292 4.07715
114.2064 4.1748
114.4836 4.22363
114.7608 4.27246
115.038 4.22363
115.3152 4.46777
115.5924 4.5166
115.8696 4.56543
116.1468 4.76074
116.424 4.90723
116.7012 4.71191
116.9784 4.66309
117.2556 4.80957
117.5328 4.95605
117.81 5.05371
118.0872 5.10254
118.3644 5.2002
118.6416 5.34668
118.9188 5.24902
119.196 5.34668
119.4732 5.34668
119.7504 5.44434
120.0276 5.49316
120.3048 5.63965
120.582 5.59082
120.8592 5.88379
121.1364 5.7373
121.4136 5.7373
121.6908 5.63965
121.968 5.7373
122.2452 5.88379
122.5224 5.93262
122.7996 6.03027
123.0768 6.12793
123.354 5.98145
123.6312 6.12793
123.9084 6.17676
124.1856 6.27441
124.4628 6.37207
124.74 6.46973
125.0172 6.37207
125.2944 6.37207
125.5716 6.32324
125.8488 6.46973
126.126 6.51855
126.4032 6.66504
126.6804 6.7627
126.9576 6.66504
127.2348 6.51855
127.512 6.81152
127.7892 6.7627
128.0664 7.00684
128.3436 6.86035
128.6208 7.10449
128.898 6.61621
129.1752 6.7627
129.4524 6.86035
129.7296 7.20215
130.0068 7.05566
130.284 6.95801
130.5612 6.95801
130.8384 7.10449
131.1156 7.10449
131.3928 7.15332
131.67 7.34863
131.9472 7.34863
132.2244 7.15332
132.5016 7.44629
132.7788 7.34863
133.056 7.25098
133.3332 7.44629
133.6104 7.34863
133.8876 7.39746
134.1648 7.34863
134.442 7.54395
134.7192 7.54395
134.9964 7.54395
135.2736 7.54395
135.5508 7.54395
135.828 7.78809
136.1052 7.73926
136.3824 7.88574
136.6596 7.78809
136.9368 7.78809
137.214 7.73926
137.4912 7.78809
137.7684 7.83691
138.0456 7.9834
138.3228 7.73926
138.6 7.83691
138.8772 8.03223
139.1544 8.3252
139.4316 8.08105
139.7088 7.9834
139.986 8.12988
140.2632 8.12988
140.5404 8.03223
140.8176 8.22754
141.0948 8.12988
141.372 7.9834
141.6492 8.12988
141.9264 8.22754
142.2036 8.3252
142.4808 8.22754
142.758 8.08105
143.0352 8.12988
143.3124 8.12988
143.5896 8.22754
143.8668 8.27637
144.144 8.22754
144.4212 8.08105
144.6984 8.42285
144.9756 8.22754
145.2528 8.37402
145.53 8.3252
145.8072 8.17871
146.0844 8.52051
146.3616 8.56934
146.6388 8.42285
146.916 8.42285
147.1932 8.3252
147.4704 8.3252
147.7476 8.42285
148.0248 8.37402
148.302 8.22754
148.5792 8.3252
148.8564 8.37402
149.1336 8.52051
149.4108 8.3252
149.688 8.22754
149.9652 8.22754
150.2424 8.42285
150.5196 8.37402
150.7968 8.42285
151.074 8.3252
151.3512 8.17871
151.6284 8.22754
151.9056 8.52051
152.1828 8.3252
152.46 8.3252
152.7372 8.27637
153.0144 8.42285
153.2916 8.56934
153.5688 8.37402
153.846 8.3252
154.1232 8.3252
154.4004 8.3252
154.6776 8.56934
154.9548 8.52051
155.232 8.37402
155.5092 8.3252
155.7864 8.27637
156.0636 8.52051
156.3408 8.3252
156.618 8.47168
156.8952 8.22754
157.1724 8.22754
157.4496 8.37402
157.7268 8.42285
158.004 8.37402
158.2812 8.27637
158.5584 8.52051
158.8356 8.61816
159.1128 8.42285
159.39 8.42285
159.6672 8.47168
159.9444 8.08105
160.2216 8.42285
160.4988 8.56934
160.776 8.47168
161.0532 8.22754
161.3304 8.3252
161.6076 8.37402
161.8848 8.52051
162.162 8.37402
162.4392 8.3252
162.7164 8.22754
162.9936 8.22754
163.2708 8.42285
163.548 8.47168
163.8252 8.17871
164.1024 8.12988
164.3796 8.17871
164.6568 8.42285
164.934 8.37402
165.2112 8.27637
165.4884 8.08105
165.7656 8.52051
166.0428 8.52051
166.32 8.61816
166.5972 8.37402
166.8744 8.22754
167.1516 8.37402
167.4288 8.27637
167.706 8.42285
167.9832 8.37402
168.2604 8.12988
168.5376 8.27637
168.8148 8.22754
169.092 8.42285
169.3692 8.37402
169.6464 8.22754
169.9236 8.37402
170.2008 8.47168
170.478 8.3252
170.7552 8.27637
171.0324 8.37402
171.3096 8.42285
171.5868 8.47168
171.864 8.52051
172.1412 8.37402
172.4184 8.17871
172.6956 8.37402
172.9728 8.47168
173.25 8.42285
173.5272 8.47168
173.8044 8.22754
174.0816 8.17871
174.3588 8.42285
174.636 8.37402
174.9132 8.27637
175.1904 8.3252
175.4676 8.27637
175.7448 8.61816
176.022 8.42285
176.2992 8.56934
176.5764 8.37402
176.8536 8.12988
177.1308 8.37402
177.408 8.3252
177.6852 8.37402
177.9624 8.42285
178.2396 8.52051
178.5168 8.52051
178.794 8.3252
179.0712 8.42285
179.3484 8.12988
179.6256 8.17871
179.9028 8.3252
180.18 8.47168
180.4572 8.37402
180.7344 8.27637
181.0116 8.3252
181.2888 8.42285
181.566 8.42285
181.8432 8.47168
182.1204 8.37402
182.3976 8.17871
182.6748 8.42285
182.952 8.47168
183.2292 8.42285
183.5064 8.27637
183.7836 8.03223
184.0608 8.27637
184.338 8.27637
184.6152 8.42285
184.8924 8.37402
185.1696 8.22754
185.4468 8.42285
185.724 8.56934
186.0012 8.61816
186.2784 8.47168
186.5556 8.22754
186.8328 8.47168
187.11 8.17871
187.3872 8.42285
187.6644 8.37402
187.9416 8.17871
188.2188 8.3252
188.496 8.3252
188.7732 8.3252
189.0504 8.47168
189.3276 8.37402
189.6048 8.22754
189.882 8.52051
190.1592 8.37402
190.4364 8.37402
190.7136 8.22754
190.9908 8.3252
191.268 8.3252
191.5452 8.37402
191.8224 8.3252
192.0996 8.12988
192.3768 8.37402
192.654 8.27637
192.9312 8.47168
193.2084 8.52051
193.4856 8.22754
193.7628 8.22754
194.04 8.27637
194.3172 8.37402
194.5944 8.37402
194.8716 8.17871
195.1488 8.42285
195.426 8.52051
195.7032 8.56934
195.9804 8.3252
196.2576 8.37402
196.5348 8.27637
196.812 8.3252
197.0892 8.42285
197.3664 8.42285
197.6436 8.37402
197.9208 8.27637
198.198 8.42285
198.4752 8.47168
198.7524 8.52051
199.0296 8.47168
199.3068 8.12988
199.584 8.42285
199.8612 8.3252
200.1384 8.42285
200.4156 8.27637
200.6928 8.22754
200.97 8.37402
201.2472 8.56934
201.5244 8.37402
201.8016 8.27637
202.0788 8.22754
202.356 8.37402
202.6332 8.42285
202.9104 8.42285
203.1876 8.37402
203.4648 8.22754
203.742 8.37402
204.0192 8.42285
204.2964 8.42285
204.5736 8.37402
204.8508 8.22754
205.128 8.42285
205.4052 8.42285
205.6824 8.56934
205.9596 8.52051
206.2368 8.27637
206.514 8.37402
206.7912 8.37402
207.0684 8.47168
207.3456 8.27637
207.6228 8.17871
207.9 8.47168
208.1772 8.37402
208.4544 8.37402
208.7316 8.42285
209.0088 8.17871
209.286 8.17871
209.5632 8.37402
209.8404 8.52051
210.1176 8.42285
210.3948 8.52051
210.672 8.12988
210.9492 8.37402
211.2264 8.27637
211.5036 8.42285
211.7808 8.12988
212.058 8.37402
212.3352 8.42285
212.6124 8.3252
212.8896 8.52051
213.1668 8.22754
213.444 8.17871
213.7212 8.27637
213.9984 8.42285
214.2756 8.37402
214.5528 8.37402
214.83 8.3252
215.1072 8.42285
215.3844 8.56934
215.6616 8.42285
215.9388 8.27637
216.216 8.22754
216.4932 8.37402
216.7704 8.56934
217.0476 8.37402
217.3248 8.22754
217.602 8.27637
217.8792 8.47168
218.1564 8.52051
218.4336 8.42285
218.7108 8.3252
218.988 8.3252
219.2652 8.17871
219.5424 8.37402
219.8196 8.3252
220.0968 8.42285
220.374 8.27637
220.6512 8.42285
220.9284 8.47168
221.2056 8.42285
221.4828 8.3252
221.76 8.17871
222.0372 8.12988
222.3144 8.42285
222.5916 8.66699
222.8688 8.37402
223.146 8.08105
223.4232 8.27637
223.7004 8.17871
223.9776 8.27637
224.2548 8.27637
224.532 8.22754
224.8092 8.22754
225.0864 8.56934
225.3636 8.61816
225.6408 8.66699
225.918 8.17871
226.1952 8.27637
226.4724 8.42285
226.7496 8.47168
227.0268 8.42285
227.304 8.12988
227.5812 8.22754
227.8584 8.42285
228.1356 8.3252
228.4128 8.27637
228.69 8.22754
228.9672 8.3252
229.2444 8.42285
229.5216 8.56934
229.7988 8.37402
230.076 8.27637
230.3532 8.12988
230.6304 8.42285
230.9076 8.42285
};
\addplot [semithick, darkorange25512714]
table {%
0 -8.95996
0.2772000000002 -9.05762
0.554400000000399 -9.10645
0.83159999999971 -9.30176
1.10879999999991 -9.2041
1.38600000000011 -9.15527
1.66320000000031 -9.05762
1.94039999999962 -8.95996
2.21759999999982 -9.2041
2.49480000000002 -9.2041
2.77200000000022 -8.95996
3.04919999999953 -9.05762
3.32639999999973 -8.95996
3.60359999999993 -9.15527
3.88080000000013 -9.2041
4.15800000000033 -9.30176
4.43519999999964 -8.95996
4.71239999999984 -9.25293
4.98960000000004 -9.10645
5.26680000000024 -9.10645
5.54399999999955 -9.00879
5.82119999999975 -9.10645
6.09839999999995 -9.10645
6.37560000000015 -9.15527
6.65280000000035 -9.10645
6.92999999999966 -9.05762
7.20719999999986 -9.10645
7.48440000000006 -9.15527
7.76160000000026 -9.10645
8.03879999999957 -9.2041
8.31599999999977 -8.95996
8.59319999999997 -9.15527
8.87040000000017 -8.95996
9.14760000000037 -9.05762
9.42479999999968 -9.10645
9.70199999999988 -9.15527
9.97920000000008 -9.25293
10.2564000000003 -9.15527
10.5335999999996 -9.25293
10.8107999999998 -9.2041
11.088 -9.10645
11.3652000000002 -9.10645
11.6424000000004 -9.00879
11.9195999999997 -9.10645
12.1967999999999 -9.10645
12.4740000000001 -9.2041
12.7512000000003 -9.15527
13.0283999999996 -9.10645
13.3055999999998 -8.95996
13.5828 -9.2041
13.8600000000002 -9.05762
14.1371999999995 -9.00879
14.4143999999997 -9.00879
14.6915999999999 -9.05762
14.9688000000001 -9.10645
15.2460000000003 -8.95996
15.5231999999996 -9.05762
15.8003999999998 -9.2041
16.0776 -9.10645
16.3548000000002 -9.15527
16.6319999999995 -9.05762
16.9091999999997 -9.2041
17.1863999999999 -9.10645
17.4636000000001 -9.10645
17.7408000000003 -9.10645
18.0179999999996 -9.25293
18.2951999999998 -9.10645
18.5724 -9.25293
18.8496000000002 -9.2041
19.1267999999996 -9.05762
19.4039999999998 -9.25293
19.6812 -9.00879
19.9584000000002 -8.95996
20.2356000000004 -9.25293
20.5127999999997 -9.05762
20.7899999999999 -9.25293
21.0672000000001 -9.30176
21.3444000000003 -9.10645
21.6215999999996 -9.10645
21.8987999999998 -9.15527
22.176 -9.2041
22.4532000000002 -9.10645
22.7304000000004 -9.15527
23.0075999999997 -9.00879
23.2847999999999 -9.10645
23.5620000000001 -9.10645
23.8392000000003 -9.10645
24.1163999999996 -9.15527
24.3935999999998 -9.2041
24.6708 -9.15527
24.9480000000002 -9.15527
25.2252000000004 -9.15527
25.5023999999997 -9.10645
25.7795999999999 -8.95996
26.0568000000001 -9.25293
26.3340000000003 -9.10645
26.6111999999996 -9.10645
26.8883999999998 -9.25293
27.1656 -9.2041
27.4428000000002 -9.15527
27.7199999999995 -8.95996
27.9971999999997 -9.10645
28.2743999999999 -9.15527
28.5516000000001 -9.10645
28.8288000000003 -9.2041
29.1059999999996 -9.00879
29.3831999999998 -9.10645
29.6604 -9.05762
29.9376000000002 -9.30176
30.2147999999995 -9.25293
30.4919999999997 -9.00879
30.7691999999999 -9.2041
31.0464000000001 -9.10645
31.3236000000003 -9.10645
31.6007999999997 -9.10645
31.8779999999999 -8.95996
32.1552000000001 -9.10645
32.4324000000002 -9.2041
32.7095999999996 -9.05762
32.9867999999998 -9.25293
33.264 -9.10645
33.5412000000002 -9.30176
33.8184000000004 -8.95996
34.0955999999997 -9.10645
34.3727999999999 -9.00879
34.6500000000001 -9.25293
34.9272000000003 -9.10645
35.2043999999996 -9.15527
35.4815999999998 -9.15527
35.7588 -9.10645
36.0360000000002 -9.10645
36.3132000000004 -9.25293
36.5903999999997 -9.15527
36.8675999999999 -9.15527
37.1448000000001 -9.05762
37.4220000000003 -9.15527
37.6991999999996 -9.10645
37.9763999999998 -8.95996
38.2536 -9.2041
38.5308000000002 -9.05762
38.8080000000004 -9.15527
39.0851999999997 -9.15527
39.3623999999999 -9.2041
39.6396000000001 -8.95996
39.9168000000003 -9.2041
40.1939999999996 -9.05762
40.4711999999998 -9.05762
40.7484 -9.00879
41.0256000000002 -9.15527
41.3027999999995 -9.05762
41.5799999999997 -9.05762
41.8571999999999 -8.95996
42.1344000000001 -8.95996
42.4116000000003 -9.15527
42.6887999999996 -9.10645
42.9659999999998 -9.10645
43.2432 -8.95996
43.5204000000002 -9.10645
43.7975999999995 -8.95996
44.0747999999997 -9.10645
44.3519999999999 -9.10645
44.6292000000001 -9.15527
44.9064000000003 -8.95996
45.1835999999997 -9.05762
45.4607999999999 -9.10645
45.7380000000001 -9.05762
46.0152000000003 -9.10645
46.2923999999996 -9.10645
46.5695999999998 -9.10645
46.8468 -9.05762
47.1240000000002 -9.15527
47.4012000000004 -9.10645
47.6783999999997 -9.15527
47.9555999999999 -9.10645
48.2328000000001 -9.00879
48.5100000000003 -9.00879
48.7871999999996 -9.10645
49.0643999999998 -9.15527
49.3416 -9.10645
49.6188000000002 -9.00879
49.8960000000004 -9.15527
50.1731999999997 -9.25293
50.4503999999999 -9.2041
50.7276000000001 -9.10645
51.0048000000003 -9.15527
51.2819999999996 -9.00879
51.5591999999998 -9.05762
51.8364 -9.2041
52.1136000000002 -9.25293
52.3907999999995 -9.25293
52.6679999999997 -9.10645
52.9451999999999 -9.15527
53.2224000000001 -8.95996
53.4996000000003 -9.15527
53.7767999999996 -9.25293
54.0539999999998 -9.15527
54.3312 -9.10645
54.6084000000002 -9.05762
54.8855999999995 -9.15527
55.1627999999997 -9.10645
55.4399999999999 -9.2041
55.7172000000001 -9.2041
55.9944000000003 -9.10645
56.2715999999996 -9.15527
56.5487999999998 -9.10645
56.826 -9.15527
57.1032000000002 -9.10645
57.3803999999996 -9.2041
57.6575999999998 -9.15527
57.9348 -9.2041
58.2120000000002 -9.05762
58.4892000000004 -9.2041
58.7663999999997 -9.10645
59.0435999999999 -9.2041
59.3208000000001 -9.05762
59.5980000000003 -9.15527
59.8751999999996 -9.10645
60.1523999999998 -8.95996
60.4296 -9.10645
60.7068000000002 -9.15527
60.9840000000004 -9.10645
61.2611999999997 -9.2041
61.5383999999999 -9.10645
61.8156000000001 -9.15527
62.0928000000003 -9.2041
62.3699999999996 -9.10645
62.6471999999998 -9.10645
62.9244 -9.30176
63.2016000000002 -9.00879
63.4788000000004 -9.05762
63.7559999999997 -9.30176
64.0331999999999 -9.2041
64.3104000000001 -9.10645
64.5876000000003 -9.15527
64.8647999999996 -9.15527
65.1419999999998 -9.2041
65.4192 -9.05762
65.6964000000002 -9.00879
65.9735999999995 -9.10645
66.2507999999997 -9.2041
66.5279999999999 -9.10645
66.8052000000001 -9.10645
67.0824000000003 -9.05762
67.3595999999996 -9.15527
67.6367999999998 -9.2041
67.914 -9.25293
68.1912000000002 -8.95996
68.4683999999995 -9.15527
68.7455999999997 -9.2041
69.0227999999999 -9.15527
69.3000000000001 -8.95996
69.5772000000003 -9.15527
69.8543999999997 -8.95996
70.1315999999998 -9.10645
70.4088 -9.15527
70.6860000000002 -9.15527
70.9631999999996 -9.10645
71.2403999999998 -9.10645
71.5176 -9.05762
71.7948000000002 -9.25293
72.0720000000004 -9.05762
72.3491999999997 -9.10645
72.6263999999999 -9.2041
72.9036000000001 -9.05762
73.1808000000003 -9.15527
73.4579999999996 -9.15527
73.7351999999998 -9.25293
74.0124 -9.10645
74.2896000000002 -9.25293
74.5668000000004 -9.05762
74.8439999999997 -9.2041
75.1211999999999 -9.15527
75.3984000000001 -9.10645
75.6756000000003 -9.10645
75.9527999999996 -9.10645
76.2299999999998 -9.2041
76.5072 -9.10645
76.7844000000002 -9.05762
77.0616000000004 -9.10645
77.3387999999997 -9.05762
77.6159999999999 -9.10645
77.8932000000001 -9.00879
78.1704000000003 -9.10645
78.4475999999996 8.91113
78.7247999999998 9.05762
79.002 9.15527
79.2792000000002 9.15527
79.5563999999995 9.00879
79.8335999999997 9.10645
80.1107999999999 9.05762
80.3880000000001 9.10645
80.6652000000003 9.00879
80.9423999999996 9.00879
81.2195999999998 9.15527
81.4968 9.30176
81.7740000000002 9.15527
82.0511999999995 9.15527
82.3283999999997 9.15527
82.6056 9.25293
82.8828000000001 9.2041
83.1600000000003 8.8623
83.4371999999996 9.00879
83.7143999999999 9.10645
83.9916000000001 9.05762
84.2688000000003 8.95996
84.5459999999996 9.10645
84.8231999999998 9.10645
85.1004 9.10645
85.3776000000002 9.00879
85.6548000000004 9.05762
85.9319999999997 9.2041
86.2091999999999 8.91113
86.4864000000001 9.10645
86.7636000000003 9.10645
87.0407999999996 9.00879
87.3179999999998 9.00879
87.5952 9.05762
87.8724000000002 9.25293
88.1496000000004 9.05762
88.4267999999997 9.10645
88.7039999999999 9.10645
88.9812000000001 8.95996
89.2584000000003 9.05762
89.5355999999996 9.00879
89.8127999999998 9.10645
90.09 9.10645
90.3672000000002 9.05762
90.6443999999995 9.00879
90.9215999999997 9.2041
91.1987999999999 8.95996
91.4760000000001 9.10645
91.7532000000003 9.25293
92.0303999999996 9.10645
92.3075999999998 9.05762
92.5848 9.10645
92.8620000000002 9.10645
93.1391999999995 9.05762
93.4163999999997 9.10645
93.6935999999999 9.10645
93.9708000000001 8.91113
94.2480000000003 8.95996
94.5251999999996 8.91113
94.8023999999998 9.15527
95.0796 9.10645
95.3568000000002 9.05762
95.6339999999996 9.05762
95.9111999999998 9.10645
96.1883999999999 9.05762
96.4656000000002 9.10645
96.7428000000004 9.10645
97.0199999999997 9.10645
97.2971999999999 9.10645
97.5744000000001 8.95996
97.8516000000003 9.05762
98.1287999999996 8.95996
98.4059999999998 9.10645
98.6832 9.10645
98.9604000000002 9.05762
99.2376000000004 9.05762
99.5147999999997 9.10645
99.7919999999999 8.95996
100.0692 9.05762
100.3464 8.95996
100.6236 8.95996
100.9008 9.15527
101.178 9.15527
101.4552 9.10645
101.7324 8.91113
102.0096 9.00879
102.2868 9.05762
102.564 9.05762
102.8412 9.2041
103.1184 9.15527
103.3956 9.15527
103.6728 8.95996
103.95 9.30176
104.2272 9.15527
104.5044 9.10645
104.7816 9.10645
105.0588 9.05762
105.336 9.2041
105.6132 9.05762
105.8904 9.10645
106.1676 9.00879
106.4448 9.15527
106.722 9.00879
106.9992 9.10645
107.2764 9.10645
107.5536 9.10645
107.8308 9.15527
108.108 9.05762
108.3852 9.10645
108.6624 9.10645
108.9396 9.25293
109.2168 9.10645
109.494 9.10645
109.7712 9.05762
110.0484 9.05762
110.3256 9.10645
110.6028 9.05762
110.88 9.10645
111.1572 9.10645
111.4344 9.05762
111.7116 8.95996
111.9888 9.15527
112.266 9.10645
112.5432 9.10645
112.8204 9.10645
113.0976 9.25293
113.3748 9.05762
113.652 9.2041
113.9292 9.10645
114.2064 9.10645
114.4836 8.91113
114.7608 8.95996
115.038 9.15527
115.3152 9.05762
115.5924 9.05762
115.8696 9.05762
116.1468 9.00879
116.424 9.10645
116.7012 9.10645
116.9784 9.00879
117.2556 9.10645
117.5328 9.2041
117.81 8.95996
118.0872 9.05762
118.3644 8.95996
118.6416 8.95996
118.9188 9.05762
119.196 9.00879
119.4732 9.10645
119.7504 9.10645
120.0276 8.95996
120.3048 9.15527
120.582 8.91113
120.8592 9.10645
121.1364 9.05762
121.4136 8.95996
121.6908 9.10645
121.968 9.00879
122.2452 8.95996
122.5224 9.10645
122.7996 8.95996
123.0768 9.10645
123.354 9.10645
123.6312 9.05762
123.9084 9.00879
124.1856 8.8623
124.4628 8.95996
124.74 9.15527
125.0172 9.05762
125.2944 9.2041
125.5716 8.91113
125.8488 9.10645
126.126 9.25293
126.4032 8.95996
126.6804 9.15527
126.9576 9.10645
127.2348 9.10645
127.512 9.2041
127.7892 9.25293
128.0664 9.15527
128.3436 8.95996
128.6208 9.05762
128.898 8.95996
129.1752 9.00879
129.4524 9.00879
129.7296 9.05762
130.0068 9.10645
130.284 8.91113
130.5612 9.10645
130.8384 9.30176
131.1156 9.00879
131.3928 8.95996
131.67 9.05762
131.9472 9.10645
132.2244 9.00879
132.5016 8.95996
132.7788 9.10645
133.056 9.05762
133.3332 8.95996
133.6104 8.95996
133.8876 9.05762
134.1648 9.00879
134.442 8.91113
134.7192 9.15527
134.9964 8.95996
135.2736 9.10645
135.5508 9.00879
135.828 9.05762
136.1052 8.95996
136.3824 9.05762
136.6596 9.10645
136.9368 9.05762
137.214 9.00879
137.4912 9.05762
137.7684 8.8623
138.0456 9.05762
138.3228 9.00879
138.6 9.05762
138.8772 9.05762
139.1544 9.10645
139.4316 9.10645
139.7088 8.95996
139.986 9.10645
140.2632 8.95996
140.5404 9.10645
140.8176 9.10645
141.0948 9.10645
141.372 8.95996
141.6492 9.05762
141.9264 9.2041
142.2036 9.00879
142.4808 9.15527
142.758 9.00879
143.0352 9.2041
143.3124 9.00879
143.5896 9.10645
143.8668 9.00879
144.144 9.00879
144.4212 9.10645
144.6984 8.95996
144.9756 9.25293
145.2528 9.10645
145.53 8.95996
145.8072 9.15527
146.0844 9.00879
146.3616 9.25293
146.6388 9.10645
146.916 9.10645
147.1932 9.00879
147.4704 9.10645
147.7476 9.10645
148.0248 8.8623
148.302 9.05762
148.5792 9.10645
148.8564 9.05762
149.1336 9.10645
149.4108 9.00879
149.688 9.10645
149.9652 8.95996
150.2424 9.00879
150.5196 9.10645
150.7968 8.95996
151.074 9.10645
151.3512 9.2041
151.6284 9.05762
151.9056 9.10645
152.1828 9.10645
152.46 9.15527
152.7372 9.15527
153.0144 8.95996
153.2916 9.00879
153.5688 9.10645
153.846 8.95996
154.1232 9.10645
154.4004 9.15527
154.6776 9.05762
154.9548 9.05762
155.232 9.00879
155.5092 9.10645
155.7864 9.10645
156.0636 9.05762
156.3408 8.8623
156.618 9.00879
156.8952 9.10645
157.1724 9.05762
157.4496 9.30176
157.7268 9.10645
158.004 9.00879
158.2812 9.05762
158.5584 9.10645
158.8356 9.00879
159.1128 9.10645
159.39 9.15527
159.6672 8.95996
159.9444 9.05762
160.2216 9.10645
160.4988 8.95996
160.776 9.15527
161.0532 8.8623
161.3304 9.10645
161.6076 8.91113
161.8848 9.10645
162.162 9.2041
162.4392 9.00879
162.7164 9.00879
162.9936 9.00879
163.2708 8.95996
163.548 9.2041
163.8252 9.05762
164.1024 9.10645
164.3796 8.91113
164.6568 9.25293
164.934 8.95996
165.2112 9.10645
165.4884 9.30176
165.7656 9.15527
166.0428 9.00879
166.32 9.00879
166.5972 9.10645
166.8744 8.91113
167.1516 9.15527
167.4288 9.00879
167.706 9.10645
167.9832 9.10645
168.2604 9.00879
168.5376 9.10645
168.8148 9.10645
169.092 9.10645
169.3692 9.10645
169.6464 9.05762
169.9236 9.00879
170.2008 9.05762
170.478 9.05762
170.7552 9.00879
171.0324 8.91113
171.3096 9.10645
171.5868 9.00879
171.864 9.00879
172.1412 8.95996
172.4184 9.10645
172.6956 9.05762
172.9728 9.10645
173.25 8.95996
173.5272 8.95996
173.8044 9.05762
174.0816 9.10645
174.3588 9.05762
174.636 9.10645
174.9132 9.10645
175.1904 9.15527
175.4676 9.15527
175.7448 9.10645
176.022 8.95996
176.2992 9.05762
176.5764 8.91113
176.8536 9.15527
177.1308 8.95996
177.408 9.10645
177.6852 9.05762
177.9624 8.95996
178.2396 9.00879
178.5168 9.05762
178.794 9.05762
179.0712 9.05762
179.3484 9.10645
179.6256 9.10645
179.9028 9.10645
180.18 9.10645
180.4572 8.95996
180.7344 9.10645
181.0116 9.10645
181.2888 9.05762
181.566 8.95996
181.8432 8.95996
182.1204 8.95996
182.3976 9.10645
182.6748 9.00879
182.952 9.10645
183.2292 8.95996
183.5064 9.10645
183.7836 9.2041
184.0608 9.25293
184.338 9.10645
184.6152 9.10645
184.8924 8.95996
185.1696 9.05762
185.4468 9.10645
185.724 9.15527
186.0012 9.05762
186.2784 8.95996
186.5556 9.10645
186.8328 9.10645
187.11 9.15527
187.3872 9.05762
187.6644 9.00879
187.9416 8.95996
188.2188 9.15527
188.496 8.95996
188.7732 9.15527
189.0504 9.10645
189.3276 9.10645
189.6048 9.15527
189.882 8.8623
190.1592 9.10645
190.4364 9.05762
190.7136 9.10645
190.9908 9.00879
191.268 8.95996
191.5452 9.05762
191.8224 9.10645
192.0996 9.00879
192.3768 9.10645
192.654 8.95996
192.9312 9.15527
193.2084 9.10645
193.4856 9.10645
193.7628 9.00879
194.04 9.00879
194.3172 9.10645
194.5944 8.95996
194.8716 9.05762
195.1488 9.10645
195.426 8.91113
195.7032 9.10645
195.9804 8.91113
196.2576 9.25293
196.5348 9.10645
196.812 9.00879
197.0892 9.00879
197.3664 9.10645
197.6436 9.00879
197.9208 9.00879
198.198 9.00879
198.4752 9.05762
198.7524 9.25293
199.0296 9.05762
199.3068 9.15527
199.584 9.10645
199.8612 9.05762
200.1384 9.25293
200.4156 8.95996
200.6928 9.10645
200.97 9.25293
201.2472 9.05762
201.5244 9.00879
201.8016 8.81348
202.0788 9.00879
202.356 8.95996
202.6332 9.00879
202.9104 8.95996
203.1876 9.05762
203.4648 9.05762
203.742 9.05762
204.0192 9.10645
204.2964 9.05762
204.5736 8.95996
204.8508 8.8623
205.128 9.05762
205.4052 8.8623
205.6824 9.10645
205.9596 9.15527
206.2368 9.15527
206.514 9.15527
206.7912 9.05762
207.0684 9.10645
207.3456 9.15527
207.6228 9.15527
207.9 9.2041
208.1772 9.05762
208.4544 9.10645
208.7316 9.10645
209.0088 9.10645
209.286 9.10645
209.5632 9.2041
209.8404 9.2041
210.1176 9.15527
210.3948 8.91113
210.672 8.95996
210.9492 9.00879
211.2264 9.10645
211.5036 9.05762
211.7808 9.05762
212.058 9.15527
212.3352 9.10645
212.6124 9.2041
212.8896 9.10645
213.1668 9.10645
213.444 9.10645
213.7212 9.10645
213.9984 9.10645
214.2756 9.00879
214.5528 9.05762
214.83 9.00879
215.1072 9.10645
215.3844 9.10645
215.6616 9.15527
215.9388 9.2041
216.216 9.2041
216.4932 8.95996
216.7704 9.10645
217.0476 8.8623
217.3248 9.05762
217.602 9.05762
217.8792 9.10645
218.1564 9.05762
218.4336 9.15527
218.7108 9.00879
218.988 9.00879
219.2652 8.95996
219.5424 9.05762
219.8196 8.95996
220.0968 9.00879
220.374 9.05762
220.6512 9.00879
220.9284 9.10645
221.2056 9.10645
221.4828 9.10645
221.76 8.95996
222.0372 9.15527
222.3144 9.05762
222.5916 9.2041
222.8688 9.10645
223.146 9.15527
223.4232 9.10645
223.7004 9.10645
223.9776 8.95996
224.2548 8.91113
224.532 9.10645
224.8092 9.05762
225.0864 9.10645
225.3636 9.10645
225.6408 9.15527
225.918 9.05762
226.1952 9.15527
226.4724 9.05762
226.7496 8.95996
227.0268 9.10645
227.304 9.10645
227.5812 9.00879
227.8584 9.25293
228.1356 9.00879
228.4128 9.00879
228.69 9.05762
228.9672 9.15527
229.2444 9.10645
229.5216 9.05762
229.7988 9.05762
230.076 9.10645
230.3532 9.05762
230.6304 9.10645
230.9076 9.05762
};
\end{axis}

\end{tikzpicture}

    \caption{Sprangresponsen til P-hastighetsregulatoren. Data er hentet fra \cite{EksempelData}}
    \label{fig:hastighet_P_regulator}
\end{figure}

\begin{figure}[h]
    \centering
    % This file was created with tikzplotlib v0.10.1.
\begin{tikzpicture}

\definecolor{darkgray176}{RGB}{176,176,176}
\definecolor{darkorange25512714}{RGB}{255,127,14}
\definecolor{steelblue31119180}{RGB}{31,119,180}

\begin{axis}[
tick align=outside,
tick pos=left,
title={Hastighet P-regulator, derivert respons},
%legend style={at={(0.95, 0.85)}, anchor=north east},
height=\figH,
width=\figW,
x grid style={darkgray176},
xlabel={Tid [ms]},
xmin=-13.7214000000001, xmax=288.149400000001,
xtick style={color=black},
xtick={-50,0,50,100,150,200,250,300},
xticklabels={
  \(\displaystyle {\ensuremath{-}50}\),
  \(\displaystyle {0}\),
  \(\displaystyle {50}\),
  \(\displaystyle {100}\),
  \(\displaystyle {150}\),
  \(\displaystyle {200}\),
  \(\displaystyle {250}\),
  \(\displaystyle {300}\)
},
y grid style={darkgray176},
ylabel={Speningsendring [V/s]},
ymin=-148.536402583333, ymax=498.70728425,
ytick style={color=black},
ytick={-200,-100,0,100,200,300,400,500},
yticklabels={
  \(\displaystyle {\ensuremath{-}200}\),
  \(\displaystyle {\ensuremath{-}100}\),
  \(\displaystyle {0}\),
  \(\displaystyle {100}\),
  \(\displaystyle {200}\),
  \(\displaystyle {300}\),
  \(\displaystyle {400}\),
  \(\displaystyle {500}\)
}
]
%\legend{$\omega_m$}
\addplot [semithick, steelblue31119180]
table {%
0 -10.5692640692493
2.77200000000377 2.34884559884681
5.54400000000221 -2.34884559884542
8.31600000000154 17.0277777777817
11.0880000000009 -22.3119288119061
13.8600000000038 8.80699855700101
16.6320000000031 5.28487253487567
19.4040000000015 2.34836459836575
22.176 -12.9175084174906
24.9480000000037 -2.93530543530631
27.7200000000031 17.0272967773061
30.4920000000015 -15.8537758537848
33.264 1.17460317460171
36.0360000000037 12.9173881673913
38.8080000000031 4.6972101972117
41.5800000000024 -22.8991101491306
44.3519999999999 8.22029822028706
47.1240000000037 5.28439153439121
49.8960000000039 -1.17412217412354
52.6680000000024 -11.7430254930314
55.4400000000008 11.1558441558286
58.2120000000046 18.2018999519113
60.984000000003 -27.5965608465681
63.7560000000024 -0.587301587302625
66.5280000000008 20.550745550718
69.3000000000046 -18.2020202020306
72.072000000003 -1.76118326118314
74.8440000000024 96.8809523810068
77.6160000000008 375.781866281716
80.3880000000019 346.423280423142
83.160000000003 374.020562770653
85.9320000000023 351.120610870808
88.7040000000008 385.176286676256
91.476000000001 370.497354497088
94.248000000003 344.662205387289
97.0200000000023 370.497163299753
99.7920000000017 349.946848244433
102.564000000001 341.138973063618
105.336000000004 320.588985089063
108.108000000003 304.73508898526
110.880000000002 264.221380471529
113.652 248.368446368108
116.424000000004 193.174963925011
119.196000000003 184.955387205491
121.968000000002 164.991702741743
124.740000000001 132.697570947433
127.512000000004 113.908609908638
130.284000000003 89.2490379990877
133.056000000002 115.08285233288
135.828000000001 97.4677729676719
138.600000000004 50.4956709956995
141.372000000002 38.7527657527633
144.144000000003 17.614598364608
146.916000000001 -11.1558441558292
149.688000000005 9.98196248196828
152.460000000003 8.80711880712096
155.232000000002 7.63287638288093
158.004000000001 -24.660413660388
160.776000000004 6.45851370851479
163.548000000003 4.69733044733164
166.320000000002 0.587181337182037
169.092000000001 -14.6790524290474
171.864000000002 24.0734728234635
174.636000000003 -7.6327561327579
177.408000000002 -5.87193362193712
180.180000000001 -11.1558441558429
182.952000000001 28.1839826839615
185.724000000003 -8.8074795574808
188.496000000002 -4.11026936027038
191.268000000002 8.2200577200597
194.040000000001 2.93590668590386
196.812000000003 -6.4588744588732
199.584000000003 -14.0918710918795
202.356000000002 27.0094997595149
205.128 -16.4405964405735
207.900000000004 -9.98160173160396
210.672000000003 14.0918710918789
213.444000000002 -2.93566618566837
216.216 -9.39430014428712
218.988000000004 -15.2665945165985
221.760000000003 34.6422558922759
224.532000000002 -14.6784511784545
227.304000000001 1.76130351130147
230.076000000004 4.10978835979063
232.848000000002 7.63299663299599
235.620000000003 -21.7247474747598
238.392000000001 2.93554593554308
241.164000000004 22.3122895622952
243.936000000003 -26.4219576719645
246.708000000003 4.69708994709326
249.480000000001 -1.17424242424161
252.252000000004 24.0732323232384
255.024000000003 -15.8531746031794
257.796000000003 -9.98172198172711
260.568000000001 12.9176286676236
263.340000000002 12.3302068302055
266.112000000002 -25.8346560846538
268.884000000003 8.80723905724437
271.656000000001 17.6147186147167
274.428000000001 -17.6147186147117
};
\end{axis}

\end{tikzpicture}

    \caption{Derivert sprangrespons til P-regulator. Dataen har blitt lavpass filtrert. Responsen har ikke en tilhørende y-akse. Data er hentet fra \cite{EksempelData}}
    \label{fig:hastighet_P_regulator_derivert}
\end{figure}

\begin{figure}[h]
    \centering
    % This file was created with tikzplotlib v0.10.1.
% Dette er eksempel data
\begin{tikzpicture}

\definecolor{darkgray176}{RGB}{176,176,176}
\definecolor{darkorange25512714}{RGB}{255,127,14}
\definecolor{steelblue31119180}{RGB}{31,119,180}

\begin{axis}[
tick align=outside,
tick pos=left,
title={Hastighet PI-regulator},
legend pos=south east,
height=\figH,
width=\figW,
x grid style={darkgray176},
xlabel={\(\displaystyle t\) [ms]},
xmin=-16.1469, xmax=339.0849,
xtick style={color=black},
xtick={-50,0,50,100,150,200,250,300,350},
xticklabels={
  \(\displaystyle {\ensuremath{-}50}\),
  \(\displaystyle {0}\),
  \(\displaystyle {50}\),
  \(\displaystyle {100}\),
  \(\displaystyle {150}\),
  \(\displaystyle {200}\),
  \(\displaystyle {250}\),
  \(\displaystyle {300}\),
  \(\displaystyle {350}\)
},
y grid style={darkgray176},
ylabel={\(\displaystyle V\) [V]},
ymin=-10.410154, ymax=10.751954,
ytick style={color=black},
ytick={-12.5,-10,-7.5,-5,-2.5,0,2.5,5,7.5,10,12.5},
yticklabels={
  \(\displaystyle {\ensuremath{-}12.5}\),
  \(\displaystyle {\ensuremath{-}10.0}\),
  \(\displaystyle {\ensuremath{-}7.5}\),
  \(\displaystyle {\ensuremath{-}5.0}\),
  \(\displaystyle {\ensuremath{-}2.5}\),
  \(\displaystyle {0.0}\),
  \(\displaystyle {2.5}\),
  \(\displaystyle {5.0}\),
  \(\displaystyle {7.5}\),
  \(\displaystyle {10.0}\),
  \(\displaystyle {12.5}\)
}
]
\legend {Respons, Referanse}
\addplot [semithick, steelblue31119180]
table {%
0 -9.2041
0.46199999999974 -9.15527
0.923999999999925 -9.2041
1.38600000000011 -9.30176
1.84799999999985 -9.30176
2.31000000000003 -9.10645
2.77199999999977 -9.25293
3.23399999999996 -9.10645
3.69600000000014 -9.30176
4.15799999999988 -9.39941
4.62000000000007 -9.00879
5.08199999999981 -9.15527
5.54399999999999 -9.30176
6.00600000000018 -9.10645
6.46799999999992 -9.44824
6.9300000000001 -9.25293
7.39199999999984 -8.8623
7.85400000000003 -9.05762
8.31599999999977 -9.10645
8.77799999999995 -9.25293
9.24000000000014 -9.30176
9.70199999999988 -9.00879
10.1640000000001 -9.05762
10.6259999999998 -9.05762
11.088 -9.10645
11.5500000000002 -9.2041
12.0119999999999 -9.39941
12.4740000000001 -9.05762
12.9359999999998 -9.30176
13.398 -9.05762
13.8599999999998 -9.15527
14.3219999999999 -9.30176
14.7840000000001 -9.05762
15.2459999999999 -9.15527
15.7080000000001 -9.30176
16.1699999999998 -8.76465
16.632 -9.00879
17.0940000000002 -9.15527
17.5559999999999 -8.95996
18.0180000000001 -9.15527
18.4799999999998 -9.15527
18.942 -9.15527
19.4039999999998 -9.25293
19.8659999999999 -9.10645
20.3280000000001 -9.2041
20.7899999999999 -9.10645
21.252 -9.05762
21.7139999999998 -9.2041
22.176 -9.2041
22.6380000000002 -9.10645
23.0999999999999 -9.2041
23.5620000000001 -9.05762
24.0239999999998 -9.00879
24.486 -9.10645
24.9479999999997 -9.00879
25.4099999999999 -8.95996
25.8720000000001 -9.05762
26.3339999999999 -8.95996
26.796 -9.15527
27.2579999999998 -9.2041
27.72 -9.10645
28.1820000000002 -9.25293
28.6439999999999 -9.15527
29.1060000000001 -9.15527
29.5679999999998 -9.25293
30.03 -9.30176
30.4919999999997 -9.30176
30.9539999999999 -9.35059
31.4160000000001 -9.25293
31.8779999999999 -9.30176
32.34 -9.30176
32.8019999999998 -9.00879
33.264 -9.25293
33.7260000000001 -9.10645
34.1879999999999 -8.81348
34.6500000000001 -9.05762
35.1119999999998 -9.10645
35.574 -9.15527
36.0360000000002 -9.44824
36.4979999999999 -9.10645
36.9600000000001 -9.25293
37.4219999999998 -9.30176
37.884 -9.15527
38.3459999999998 -9.30176
38.808 -9.2041
39.2700000000001 -9.10645
39.7319999999999 -9.35059
40.1940000000001 -9.30176
40.6559999999998 -9.10645
41.118 -9.25293
41.5800000000002 -9.15527
42.0419999999999 -9.05762
42.5040000000001 -9.30176
42.9659999999998 -9.00879
43.428 -9.05762
43.8899999999998 -9.00879
44.3519999999999 -8.91113
44.8140000000001 -9.15527
45.2759999999999 -9.15527
45.7380000000001 -9.05762
46.1999999999998 -9.25293
46.662 -9.15527
47.1240000000002 -9.25293
47.5859999999999 -9.25293
48.0480000000001 -9.15527
48.5099999999998 -9.15527
48.972 -9.39941
49.4339999999998 -8.95996
49.8959999999999 -9.2041
50.3580000000001 -9.2041
50.8199999999999 -8.95996
51.2820000000001 -9.30176
51.7439999999998 -9.25293
52.206 -8.91113
52.6680000000002 -9.25293
53.1299999999999 -9.00879
53.5920000000001 -9.10645
54.0539999999998 -9.10645
54.516 -9.15527
54.9779999999998 -9.25293
55.4399999999999 -9.35059
55.9020000000001 -8.95996
56.3639999999999 -9.39941
56.826 -9.30176
57.2879999999998 -9.00879
57.75 -9.30176
58.2120000000002 -9.35059
58.6739999999999 -9.30176
59.1360000000001 -9.25293
59.5979999999998 -9.05762
60.06 -9.44824
60.5219999999997 -9.25293
60.9839999999999 -9.00879
61.4460000000001 -9.30176
61.9079999999999 -9.2041
62.37 -8.95996
62.8319999999998 -9.15527
63.294 -9.10645
63.7560000000001 -9.10645
64.2179999999999 -9.15527
64.6800000000001 -9.15527
65.1419999999998 -9.10645
65.604 -9.30176
66.0660000000002 -9.30176
66.5279999999999 -9.35059
66.9900000000001 -9.30176
67.4519999999998 -9.00879
67.914 -9.30176
68.3759999999998 -9.35059
68.838 -8.95996
69.3000000000001 -9.30176
69.7619999999999 -9.15527
70.2240000000001 -8.71582
70.6859999999998 -8.66699
71.148 -8.42285
71.6100000000002 -8.22754
72.0719999999999 -8.3252
72.5340000000001 -7.88574
72.9959999999998 -7.88574
73.458 -7.78809
73.9199999999998 -7.69043
74.3819999999999 -7.25098
74.8440000000001 -7.25098
75.3059999999999 -6.95801
75.7680000000001 -6.90918
76.2299999999998 -6.46973
76.692 -6.61621
77.1540000000002 -6.12793
77.6159999999999 -6.0791
78.0780000000001 -5.88379
78.5399999999998 -5.88379
79.002 -5.83496
79.4639999999998 -5.44434
79.9259999999999 -5.2002
80.3880000000001 -5.15137
80.8499999999999 -5.00488
81.3120000000001 -4.8584
81.7739999999998 -4.66309
82.236 -4.41895
82.6980000000002 -4.12598
83.1599999999999 -3.88184
83.6220000000001 -3.93066
84.0839999999998 -3.73535
84.546 -3.49121
85.0079999999998 -3.44238
85.4699999999999 -2.9541
85.9320000000001 -2.9541
86.3939999999999 -2.85645
86.8560000000001 -2.9541
87.3179999999998 -2.56348
87.78 -2.31934
88.2420000000002 -2.31934
88.7039999999999 -2.0752
89.1660000000001 -1.92871
89.6279999999998 -1.68457
90.09 -1.48926
90.5519999999997 -1.3916
91.0139999999999 -1.0498
91.4760000000001 -0.90332
91.9379999999998 -0.756836
92.4 -0.805664
92.8619999999998 -0.610352
93.324 -0.366211
93.7860000000001 -0.170898
94.2479999999999 -0.170898
94.7100000000001 0.170898
95.1719999999998 0.219727
95.634 0.415039
96.0960000000002 0.561523
96.5579999999999 0.805664
97.0200000000001 0.756836
97.4819999999998 0.90332
97.944 1.24512
98.4059999999998 1.44043
98.868 1.44043
99.3300000000001 1.7334
99.7919999999999 1.97754
100.254 1.97754
100.716 2.17285
101.178 2.31934
101.64 2.6123
102.102 2.66113
102.564 2.80762
103.026 2.85645
103.488 2.80762
103.95 3.14941
104.412 3.2959
104.874 3.39355
105.336 3.54004
105.798 3.6377
106.26 3.88184
106.722 3.88184
107.184 4.12598
107.646 4.12598
108.108 4.37012
108.57 4.56543
109.032 4.56543
109.494 4.5166
109.956 4.80957
110.418 5.00488
110.88 5.00488
111.342 5.24902
111.804 5.24902
112.266 5.2002
112.728 5.68848
113.19 5.49316
113.652 5.63965
114.114 5.54199
114.576 5.78613
115.038 5.98145
115.5 6.03027
115.962 5.93262
116.424 6.12793
116.886 6.27441
117.348 6.27441
117.81 6.32324
118.272 6.4209
118.734 6.56738
119.196 6.66504
119.658 6.4209
120.12 6.71387
120.582 6.71387
121.044 6.7627
121.506 6.66504
121.968 7.10449
122.43 7.20215
122.892 7.10449
123.354 7.15332
123.816 7.25098
124.278 7.44629
124.74 7.10449
125.202 7.34863
125.664 7.49512
126.126 7.2998
126.588 7.39746
127.05 7.49512
127.512 7.54395
127.974 7.54395
128.436 7.69043
128.898 7.88574
129.36 7.69043
129.822 8.03223
130.284 7.73926
130.746 7.78809
131.208 7.78809
131.67 7.93457
132.132 7.93457
132.594 8.12988
133.056 8.12988
133.518 8.08105
133.98 8.12988
134.442 8.12988
134.904 8.22754
135.366 8.22754
135.828 8.3252
136.29 8.37402
136.752 8.3252
137.214 8.22754
137.676 8.42285
138.138 8.37402
138.6 8.52051
139.062 8.47168
139.524 8.27637
139.986 8.52051
140.448 8.52051
140.91 8.37402
141.372 8.66699
141.834 8.61816
142.296 8.22754
142.758 8.61816
143.22 8.76465
143.682 8.66699
144.144 8.91113
144.606 8.56934
145.068 8.66699
145.53 8.8623
145.992 8.76465
146.454 8.71582
146.916 8.61816
147.378 8.81348
147.84 8.81348
148.302 9.00879
148.764 8.8623
149.226 8.76465
149.688 8.91113
150.15 8.71582
150.612 8.8623
151.074 8.91113
151.536 8.81348
151.998 9.2041
152.46 9.15527
152.922 8.8623
153.384 9.10645
153.846 9.10645
154.308 8.95996
154.77 9.35059
155.232 9.10645
155.694 9.10645
156.156 9.10645
156.618 9.05762
157.08 9.10645
157.542 9.15527
158.004 8.95996
158.466 9.25293
158.928 9.15527
159.39 9.05762
159.852 9.15527
160.314 9.15527
160.776 8.95996
161.238 9.49707
161.7 9.30176
162.162 9.25293
162.624 9.30176
163.086 9.00879
163.548 9.30176
164.01 9.2041
164.472 8.95996
164.934 9.35059
165.396 9.30176
165.858 9.2041
166.32 9.39941
166.782 9.2041
167.244 9.2041
167.706 9.30176
168.168 9.15527
168.63 9.30176
169.092 9.30176
169.554 9.10645
170.016 9.35059
170.478 9.44824
170.94 9.25293
171.402 9.39941
171.864 9.15527
172.326 9.49707
172.788 9.35059
173.25 9.35059
173.712 9.35059
174.174 9.44824
174.636 9.2041
175.098 9.35059
175.56 9.39941
176.022 9.00879
176.484 9.39941
176.946 9.2041
177.408 9.15527
177.87 9.35059
178.332 9.15527
178.794 9.59473
179.256 9.44824
179.718 9.35059
180.18 9.59473
180.642 9.35059
181.104 9.44824
181.566 9.44824
182.028 9.30176
182.49 9.64355
182.952 9.49707
183.414 9.39941
183.876 9.59473
184.338 9.44824
184.8 9.49707
185.262 9.49707
185.724 9.39941
186.186 9.49707
186.648 9.30176
187.11 9.35059
187.572 9.5459
188.034 9.49707
188.496 9.30176
188.958 9.64355
189.42 9.30176
189.882 9.39941
190.344 9.64355
190.806 9.44824
191.268 9.64355
191.73 9.49707
192.192 9.30176
192.654 9.59473
193.116 9.35059
193.578 9.35059
194.04 9.5459
194.502 9.30176
194.964 9.49707
195.426 9.39941
195.888 9.35059
196.35 9.59473
196.812 9.39941
197.274 9.49707
197.736 9.79004
198.198 9.49707
198.66 9.74121
199.122 9.64355
199.584 9.39941
200.046 9.59473
200.508 9.49707
200.97 9.44824
201.432 9.5459
201.894 9.30176
202.356 9.39941
202.818 9.49707
203.28 9.44824
203.742 9.64355
204.204 9.39941
204.666 9.44824
205.128 9.59473
205.59 9.59473
206.052 9.74121
206.514 9.64355
206.976 9.44824
207.438 9.64355
207.9 9.49707
208.362 9.49707
208.824 9.39941
209.286 9.2041
209.748 9.10645
210.21 9.15527
210.672 9.05762
211.134 8.95996
211.596 9.05762
212.058 8.76465
212.52 9.00879
212.982 9.10645
213.444 9.05762
213.906 9.2041
214.368 9.15527
214.83 9.10645
215.292 9.15527
215.754 9.30176
216.216 9.2041
216.678 9.10645
217.14 9.00879
217.602 9.15527
218.064 9.10645
218.526 9.00879
218.988 9.30176
219.45 9.15527
219.912 9.00879
220.374 9.2041
220.836 8.91113
221.298 9.25293
221.76 9.25293
222.222 9.05762
222.684 9.39941
223.146 9.35059
223.608 8.95996
224.07 9.25293
224.532 9.10645
224.994 9.15527
225.456 9.10645
225.918 9.15527
226.38 9.05762
226.842 9.00879
227.304 9.25293
227.766 9.10645
228.228 9.25293
228.69 8.81348
229.152 9.10645
229.614 8.95996
230.076 8.8623
230.538 9.05762
231 9.25293
231.462 9.05762
231.924 9.35059
232.386 9.30176
232.848 9.25293
233.31 9.25293
233.772 9.10645
234.234 9.35059
234.696 9.30176
235.158 8.95996
235.62 9.00879
236.082 9.15527
236.544 9.10645
237.006 9.15527
237.468 9.00879
237.93 9.15527
238.392 9.30176
238.854 8.91113
239.316 8.95996
239.778 9.15527
240.24 9.05762
240.702 9.39941
241.164 9.35059
241.626 9.15527
242.088 9.30176
242.55 8.95996
243.012 9.30176
243.474 9.25293
243.936 9.00879
244.398 9.00879
244.86 9.2041
245.322 9.05762
245.784 9.05762
246.246 9.2041
246.708 9.10645
247.17 9.30176
247.632 9.39941
248.094 8.95996
248.556 9.2041
249.018 9.05762
249.48 8.8623
249.942 9.35059
250.404 9.30176
250.866 9.30176
251.328 9.25293
251.79 9.25293
252.252 9.10645
252.714 9.39941
253.176 8.95996
253.638 9.2041
254.1 9.30176
254.562 9.10645
255.024 9.00879
255.486 9.10645
255.948 9.10645
256.41 9.10645
256.872 9.10645
257.334 9.05762
257.796 9.10645
258.258 8.91113
258.72 9.10645
259.182 9.35059
259.644 9.00879
260.106 9.30176
260.568 9.30176
261.03 9.10645
261.492 9.39941
261.954 9.15527
262.416 9.2041
262.878 9.35059
263.34 8.95996
263.802 9.10645
264.264 9.25293
264.726 8.91113
265.188 9.25293
265.65 9.05762
266.112 9.10645
266.574 9.05762
267.036 9.00879
267.498 9.00879
267.96 9.10645
268.422 9.2041
268.884 9.10645
269.346 9.25293
269.808 9.15527
270.27 9.2041
270.732 9.25293
271.194 8.95996
271.656 9.30176
272.118 9.15527
272.58 9.10645
273.042 9.10645
273.504 9.25293
273.966 9.10645
274.428 9.10645
274.89 9.10645
275.352 9.00879
275.814 9.15527
276.276 8.76465
276.738 9.2041
277.2 9.25293
277.662 9.00879
278.124 9.2041
278.586 9.30176
279.048 9.10645
279.51 9.25293
279.972 9.05762
280.434 9.10645
280.896 9.10645
281.358 9.05762
281.82 9.15527
282.282 9.15527
282.744 8.95996
283.206 9.15527
283.668 9.15527
284.13 8.95996
284.592 9.15527
285.054 9.2041
285.516 9.15527
285.978 9.39941
286.44 9.10645
286.902 9.10645
287.364 9.15527
287.826 9.2041
288.288 9.30176
288.75 9.25293
289.212 9.00879
289.674 9.2041
290.136 9.2041
290.598 9.15527
291.06 9.15527
291.522 9.10645
291.984 9.2041
292.446 9.15527
292.908 9.05762
293.37 9.05762
293.832 9.15527
294.294 8.91113
294.756 9.30176
295.218 9.30176
295.68 8.8623
296.142 9.2041
296.604 9.2041
297.066 9.2041
297.528 9.25293
297.99 9.05762
298.452 8.95996
298.914 9.2041
299.376 8.95996
299.838 9.05762
300.3 9.15527
300.762 9.05762
301.224 9.15527
301.686 9.2041
302.148 9.05762
302.61 9.10645
303.072 9.05762
303.534 9.00879
303.996 9.25293
304.458 9.30176
304.92 9.2041
305.382 9.10645
305.844 9.05762
306.306 9.30176
306.768 9.44824
307.23 8.95996
307.692 9.10645
308.154 9.00879
308.616 9.00879
309.078 9.2041
309.54 8.91113
310.002 9.05762
310.464 9.15527
310.926 9.10645
311.388 8.95996
311.85 9.25293
312.312 9.15527
312.774 9.2041
313.236 9.15527
313.698 9.00879
314.16 9.25293
314.622 9.30176
315.084 9.05762
315.546 9.39941
316.008 8.95996
316.47 8.95996
316.932 9.25293
317.394 9.00879
317.856 9.10645
318.318 9.15527
318.78 8.8623
319.242 9.00879
319.704 9.30176
320.166 9.00879
320.628 9.15527
321.09 9.15527
321.552 8.91113
322.014 9.35059
322.476 9.30176
322.938 8.95996
};
\addplot [semithick, darkorange25512714]
table {%
0 -9.05762
0.46199999999974 -8.95996
0.923999999999925 -9.10645
1.38600000000011 -9.15527
1.84799999999985 -9.10645
2.31000000000003 -9.15527
2.77199999999977 -8.95996
3.23399999999996 -9.10645
3.69600000000014 -9.25293
4.15799999999988 -8.95996
4.62000000000007 -9.05762
5.08199999999981 -9.15527
5.54399999999999 -8.95996
6.00600000000018 -9.15527
6.46799999999992 -9.15527
6.9300000000001 -8.95996
7.39199999999984 -9.10645
7.85400000000003 -9.30176
8.31599999999977 -9.15527
8.77799999999995 -9.15527
9.24000000000014 -9.25293
9.70199999999988 -9.15527
10.1640000000001 -8.95996
10.6259999999998 -9.00879
11.088 -8.95996
11.5500000000002 -9.25293
12.0119999999999 -9.10645
12.4740000000001 -9.2041
12.9359999999998 -9.10645
13.398 -9.15527
13.8599999999998 -9.15527
14.3219999999999 -9.2041
14.7840000000001 -9.10645
15.2459999999999 -9.10645
15.7080000000001 -8.95996
16.1699999999998 -9.15527
16.632 -9.10645
17.0940000000002 -9.10645
17.5559999999999 -9.15527
18.0180000000001 -8.95996
18.4799999999998 -9.15527
18.942 -9.00879
19.4039999999998 -9.2041
19.8659999999999 -9.10645
20.3280000000001 -9.2041
20.7899999999999 -8.95996
21.252 -9.05762
21.7139999999998 -9.2041
22.176 -9.10645
22.6380000000002 -9.15527
23.0999999999999 -9.2041
23.5620000000001 -9.05762
24.0239999999998 -9.00879
24.486 -9.2041
24.9479999999997 -9.05762
25.4099999999999 -9.05762
25.8720000000001 -9.10645
26.3339999999999 -9.10645
26.796 -9.10645
27.2579999999998 -9.15527
27.72 -9.25293
28.1820000000002 -9.30176
28.6439999999999 -9.10645
29.1060000000001 -9.00879
29.5679999999998 -9.10645
30.03 -9.10645
30.4919999999997 -9.2041
30.9539999999999 -9.25293
31.4160000000001 -9.05762
31.8779999999999 -9.10645
32.34 -9.05762
32.8019999999998 -9.15527
33.264 -9.25293
33.7260000000001 -9.10645
34.1879999999999 -9.10645
34.6500000000001 -9.25293
35.1119999999998 -9.25293
35.574 -9.10645
36.0360000000002 -9.10645
36.4979999999999 -9.15527
36.9600000000001 -9.10645
37.4219999999998 -9.15527
37.884 -9.10645
38.3459999999998 -9.2041
38.808 -9.10645
39.2700000000001 -9.10645
39.7319999999999 -9.05762
40.1940000000001 -9.2041
40.6559999999998 -9.15527
41.118 -9.15527
41.5800000000002 -9.2041
42.0419999999999 -9.10645
42.5040000000001 -9.10645
42.9659999999998 -9.10645
43.428 -9.15527
43.8899999999998 -9.05762
44.3519999999999 -9.25293
44.8140000000001 -9.10645
45.2759999999999 -9.10645
45.7380000000001 -9.15527
46.1999999999998 -9.25293
46.662 -9.15527
47.1240000000002 -9.10645
47.5859999999999 -9.05762
48.0480000000001 -9.15527
48.5099999999998 -9.00879
48.972 -9.10645
49.4339999999998 -9.15527
49.8959999999999 -8.95996
50.3580000000001 -9.15527
50.8199999999999 -9.10645
51.2820000000001 -9.10645
51.7439999999998 -9.10645
52.206 -9.10645
52.6680000000002 -9.10645
53.1299999999999 -9.05762
53.5920000000001 -9.10645
54.0539999999998 -9.00879
54.516 -9.10645
54.9779999999998 -9.15527
55.4399999999999 -9.10645
55.9020000000001 -9.00879
56.3639999999999 -9.00879
56.826 -9.15527
57.2879999999998 -9.10645
57.75 -9.10645
58.2120000000002 -9.10645
58.6739999999999 -9.2041
59.1360000000001 -9.25293
59.5979999999998 -9.10645
60.06 -9.10645
60.5219999999997 -9.15527
60.9839999999999 -9.10645
61.4460000000001 -9.15527
61.9079999999999 -9.10645
62.37 -9.30176
62.8319999999998 -9.25293
63.294 -9.15527
63.7560000000001 -9.10645
64.2179999999999 -9.10645
64.6800000000001 -9.10645
65.1419999999998 -9.15527
65.604 -9.10645
66.0660000000002 -9.15527
66.5279999999999 -9.15527
66.9900000000001 -9.2041
67.4519999999998 -9.25293
67.914 -9.15527
68.3759999999998 -9.25293
68.838 -9.00879
69.3000000000001 -9.10645
69.7619999999999 9.05762
70.2240000000001 9.10645
70.6859999999998 8.95996
71.148 9.10645
71.6100000000002 9.00879
72.0719999999999 8.95996
72.5340000000001 9.15527
72.9959999999998 9.2041
73.458 9.15527
73.9199999999998 9.10645
74.3819999999999 8.91113
74.8440000000001 9.10645
75.3059999999999 9.15527
75.7680000000001 9.10645
76.2299999999998 9.05762
76.692 9.25293
77.1540000000002 9.10645
77.6159999999999 9.25293
78.0780000000001 9.05762
78.5399999999998 9.10645
79.002 9.25293
79.4639999999998 9.10645
79.9259999999999 9.05762
80.3880000000001 9.00879
80.8499999999999 9.10645
81.3120000000001 9.10645
81.7739999999998 8.95996
82.236 8.95996
82.6980000000002 9.05762
83.1599999999999 9.10645
83.6220000000001 9.05762
84.0839999999998 9.15527
84.546 8.95996
85.0079999999998 9.00879
85.4699999999999 9.2041
85.9320000000001 9.10645
86.3939999999999 8.95996
86.8560000000001 9.15527
87.3179999999998 9.15527
87.78 9.25293
88.2420000000002 9.10645
88.7039999999999 9.10645
89.1660000000001 9.00879
89.6279999999998 9.00879
90.09 9.05762
90.5519999999997 9.10645
91.0139999999999 9.10645
91.4760000000001 9.25293
91.9379999999998 9.05762
92.4 8.95996
92.8619999999998 9.00879
93.324 9.05762
93.7860000000001 9.2041
94.2479999999999 9.10645
94.7100000000001 9.2041
95.1719999999998 9.10645
95.634 9.05762
96.0960000000002 9.05762
96.5579999999999 9.05762
97.0200000000001 8.95996
97.4819999999998 8.95996
97.944 9.10645
98.4059999999998 8.91113
98.868 9.10645
99.3300000000001 9.00879
99.7919999999999 9.00879
100.254 9.00879
100.716 9.10645
101.178 9.05762
101.64 9.15527
102.102 9.05762
102.564 9.05762
103.026 9.10645
103.488 9.10645
103.95 8.91113
104.412 9.05762
104.874 9.05762
105.336 9.15527
105.798 9.10645
106.26 9.10645
106.722 9.10645
107.184 9.10645
107.646 9.00879
108.108 9.05762
108.57 9.10645
109.032 9.10645
109.494 9.2041
109.956 9.05762
110.418 9.10645
110.88 9.25293
111.342 8.95996
111.804 9.15527
112.266 9.15527
112.728 9.10645
113.19 9.00879
113.652 9.25293
114.114 9.15527
114.576 9.05762
115.038 9.10645
115.5 8.95996
115.962 9.15527
116.424 9.10645
116.886 9.2041
117.348 9.00879
117.81 9.10645
118.272 8.95996
118.734 9.00879
119.196 9.10645
119.658 8.81348
120.12 9.15527
120.582 8.95996
121.044 9.05762
121.506 9.00879
121.968 9.10645
122.43 9.10645
122.892 8.95996
123.354 9.10645
123.816 9.10645
124.278 9.15527
124.74 9.05762
125.202 9.10645
125.664 9.10645
126.126 9.05762
126.588 9.05762
127.05 8.95996
127.512 9.10645
127.974 9.10645
128.436 8.8623
128.898 9.10645
129.36 9.10645
129.822 9.10645
130.284 9.10645
130.746 9.10645
131.208 9.10645
131.67 9.05762
132.132 9.10645
132.594 9.05762
133.056 9.10645
133.518 9.05762
133.98 9.10645
134.442 9.00879
134.904 9.00879
135.366 9.10645
135.828 9.10645
136.29 9.00879
136.752 9.00879
137.214 9.15527
137.676 9.2041
138.138 9.10645
138.6 9.10645
139.062 9.2041
139.524 9.05762
139.986 9.10645
140.448 9.10645
140.91 9.00879
141.372 9.2041
141.834 9.00879
142.296 9.10645
142.758 8.95996
143.22 9.10645
143.682 8.95996
144.144 9.10645
144.606 9.05762
145.068 8.95996
145.53 8.95996
145.992 9.00879
146.454 9.10645
146.916 9.05762
147.378 9.00879
147.84 8.95996
148.302 9.05762
148.764 9.10645
149.226 9.10645
149.688 9.05762
150.15 9.00879
150.612 9.10645
151.074 8.95996
151.536 9.00879
151.998 9.10645
152.46 9.05762
152.922 8.95996
153.384 8.95996
153.846 9.10645
154.308 8.95996
154.77 8.95996
155.232 9.10645
155.694 9.05762
156.156 9.05762
156.618 9.15527
157.08 9.15527
157.542 9.15527
158.004 9.05762
158.466 9.10645
158.928 9.15527
159.39 9.05762
159.852 9.10645
160.314 8.95996
160.776 8.81348
161.238 8.95996
161.7 9.10645
162.162 9.25293
162.624 9.15527
163.086 9.15527
163.548 9.15527
164.01 9.05762
164.472 9.00879
164.934 9.00879
165.396 9.05762
165.858 9.10645
166.32 9.10645
166.782 8.95996
167.244 9.25293
167.706 8.91113
168.168 9.05762
168.63 9.00879
169.092 9.15527
169.554 9.05762
170.016 8.95996
170.478 9.15527
170.94 8.95996
171.402 9.00879
171.864 9.10645
172.326 9.25293
172.788 8.95996
173.25 8.95996
173.712 9.15527
174.174 9.15527
174.636 8.95996
175.098 9.10645
175.56 9.05762
176.022 9.10645
176.484 8.95996
176.946 9.00879
177.408 9.2041
177.87 9.15527
178.332 9.05762
178.794 8.81348
179.256 9.05762
179.718 8.91113
180.18 9.00879
180.642 9.00879
181.104 9.15527
181.566 9.10645
182.028 9.05762
182.49 9.10645
182.952 9.10645
183.414 9.15527
183.876 8.95996
184.338 9.15527
184.8 9.05762
185.262 9.05762
185.724 9.15527
186.186 8.95996
186.648 9.00879
187.11 9.05762
187.572 9.15527
188.034 9.05762
188.496 9.05762
188.958 9.10645
189.42 9.10645
189.882 9.00879
190.344 9.10645
190.806 9.10645
191.268 9.10645
191.73 8.95996
192.192 9.10645
192.654 9.15527
193.116 9.10645
193.578 9.00879
194.04 9.10645
194.502 9.10645
194.964 9.15527
195.426 9.10645
195.888 9.00879
196.35 9.25293
196.812 8.95996
197.274 9.10645
197.736 9.25293
198.198 9.10645
198.66 9.25293
199.122 9.05762
199.584 9.2041
200.046 9.10645
200.508 9.05762
200.97 9.05762
201.432 8.95996
201.894 9.10645
202.356 9.00879
202.818 9.05762
203.28 9.00879
203.742 9.10645
204.204 9.10645
204.666 9.05762
205.128 9.10645
205.59 9.00879
206.052 9.10645
206.514 9.05762
206.976 9.25293
207.438 9.10645
207.9 9.00879
208.362 9.00879
208.824 9.10645
209.286 9.05762
209.748 9.00879
210.21 9.2041
210.672 9.05762
211.134 8.95996
211.596 9.00879
212.058 9.15527
212.52 9.10645
212.982 9.00879
213.444 9.25293
213.906 9.15527
214.368 9.10645
214.83 9.00879
215.292 9.05762
215.754 9.10645
216.216 8.91113
216.678 9.10645
217.14 9.15527
217.602 9.15527
218.064 8.8623
218.526 8.95996
218.988 9.05762
219.45 9.00879
219.912 9.15527
220.374 9.05762
220.836 9.00879
221.298 9.10645
221.76 9.10645
222.222 9.00879
222.684 9.10645
223.146 9.05762
223.608 9.05762
224.07 9.05762
224.532 9.30176
224.994 9.2041
225.456 9.25293
225.918 9.05762
226.38 9.05762
226.842 9.10645
227.304 9.2041
227.766 9.00879
228.228 9.10645
228.69 8.91113
229.152 9.10645
229.614 9.10645
230.076 9.05762
230.538 9.10645
231 9.10645
231.462 9.00879
231.924 9.25293
232.386 9.10645
232.848 9.15527
233.31 9.10645
233.772 9.10645
234.234 9.10645
234.696 9.39941
235.158 9.15527
235.62 9.00879
236.082 9.25293
236.544 9.10645
237.006 9.2041
237.468 9.15527
237.93 9.15527
238.392 8.95996
238.854 9.05762
239.316 9.10645
239.778 9.15527
240.24 9.10645
240.702 9.25293
241.164 9.10645
241.626 9.10645
242.088 9.00879
242.55 8.95996
243.012 9.05762
243.474 9.10645
243.936 9.25293
244.398 9.15527
244.86 8.95996
245.322 9.05762
245.784 9.10645
246.246 9.10645
246.708 9.00879
247.17 9.10645
247.632 9.25293
248.094 9.10645
248.556 9.15527
249.018 9.15527
249.48 9.05762
249.942 9.2041
250.404 9.10645
250.866 9.10645
251.328 8.95996
251.79 9.2041
252.252 8.95996
252.714 9.10645
253.176 8.95996
253.638 9.00879
254.1 9.10645
254.562 9.15527
255.024 9.00879
255.486 9.25293
255.948 8.95996
256.41 9.10645
256.872 9.10645
257.334 8.95996
257.796 9.2041
258.258 9.10645
258.72 9.25293
259.182 9.10645
259.644 9.10645
260.106 9.15527
260.568 9.05762
261.03 9.30176
261.492 9.15527
261.954 9.10645
262.416 9.00879
262.878 8.95996
263.34 9.05762
263.802 8.95996
264.264 9.10645
264.726 9.10645
265.188 9.05762
265.65 9.05762
266.112 8.95996
266.574 9.10645
267.036 9.05762
267.498 9.05762
267.96 9.00879
268.422 9.00879
268.884 9.00879
269.346 9.00879
269.808 9.10645
270.27 9.10645
270.732 9.05762
271.194 9.05762
271.656 9.10645
272.118 9.05762
272.58 8.95996
273.042 8.95996
273.504 9.05762
273.966 9.15527
274.428 9.15527
274.89 9.10645
275.352 9.10645
275.814 9.10645
276.276 8.95996
276.738 8.95996
277.2 9.10645
277.662 9.2041
278.124 9.05762
278.586 9.15527
279.048 8.95996
279.51 9.10645
279.972 8.95996
280.434 8.95996
280.896 9.15527
281.358 9.2041
281.82 9.25293
282.282 9.00879
282.744 9.10645
283.206 9.10645
283.668 9.00879
284.13 9.10645
284.592 8.95996
285.054 9.00879
285.516 9.05762
285.978 9.15527
286.44 9.05762
286.902 8.81348
287.364 9.10645
287.826 9.00879
288.288 9.10645
288.75 9.00879
289.212 9.00879
289.674 9.05762
290.136 9.05762
290.598 9.2041
291.06 9.10645
291.522 9.10645
291.984 9.2041
292.446 9.10645
292.908 9.00879
293.37 9.05762
293.832 8.95996
294.294 9.25293
294.756 9.00879
295.218 9.10645
295.68 9.10645
296.142 9.10645
296.604 9.05762
297.066 9.15527
297.528 9.10645
297.99 9.00879
298.452 9.15527
298.914 9.15527
299.376 9.10645
299.838 9.05762
300.3 9.05762
300.762 9.2041
301.224 9.10645
301.686 9.15527
302.148 9.05762
302.61 8.95996
303.072 8.95996
303.534 9.10645
303.996 9.10645
304.458 9.25293
304.92 8.95996
305.382 9.05762
305.844 9.10645
306.306 9.00879
306.768 9.05762
307.23 8.95996
307.692 9.10645
308.154 9.00879
308.616 9.10645
309.078 8.95996
309.54 9.05762
310.002 9.10645
310.464 9.05762
310.926 9.05762
311.388 9.10645
311.85 8.8623
312.312 9.10645
312.774 9.05762
313.236 9.15527
313.698 9.15527
314.16 9.15527
314.622 9.10645
315.084 9.10645
315.546 9.10645
316.008 9.2041
316.47 9.15527
316.932 9.15527
317.394 9.15527
317.856 8.95996
318.318 8.91113
318.78 9.10645
319.242 9.05762
319.704 9.00879
320.166 9.10645
320.628 9.15527
321.09 9.10645
321.552 9.10645
322.014 8.95996
322.476 9.00879
322.938 9.15527
};
\end{axis}

\end{tikzpicture}

    \caption{Sprangresponsen til PI-hastighetsregulatoren. Data er hentet fra \cite{EksempelData}}
    \label{fig:hastighet_PI_regulator}
\end{figure}

\subsection{Diskusjon}

I \autoref{fig:hastighet_P_regulator} er det et tydelig stasjonæravvik, mellom responen til regulatoren og referanseverdien. Det er flere årsaker til at stasjonværavviket oppstår, blant annet friksjon i motoren, giret, og potmeteret. Dersom motoren bremses, vil det stasjonæreavviket øke, til tross for at pådraget også vil øke. Observasjon \ref{obs:hastighet_regulator_breming_med_finger} viser også at dette stasjonæravviket vil øke dersom motoren bremses mer.

Til tross for at det er et stasjonæravvik er det ikke så stort. Dette kommer antageligvis av at $K_p$ er stor. Siden $K_p$ er stor kan dette føre til at pådraget til regulatoren går i metning. Dette vil føre til at vi får slew-rate i regulatoren. Dette kommer fram i \autoref{fig:hastighet_P_regulator_derivert}. I intervallet \SI{80}{\milli\second} til \SI{100}{\milli\second} er akselerasjonen til motoren konstant, altså er endringen til hastigheten begrenset.

PI-regulatoren oppnår stasjonærverdi, som vist i \autoref{fig:hastighet_PI_regulator} og i observasjon \ref{obs:hastighet_regulator_breming_med_finger}. Etter omtrent \SI{150}{\milli\second} er det er oversving, som viser at systemet er litt underdempet. 

% Sammenlign hvor følsom motorhastigheten er for ekstra belastning n˚ar vi bruker regulatorenkontra å styre motoren direkte slik dere gjorde i forrige uke