\section{Hastighetsregulator}\label{sec:hastighetsreg}

\begin{figure}[t]
    \centering
    \begin{circuitikz} [scale=0.5, transform shape]
    \ctikzset{resistor = european}

    % --- OP1 ---
    \node[op amp](OP1) {$OP1$};
    
    \draw (OP1.-)
    to[R, l_=$R_1$] ++(-2, 0)
    to[short, o-, l=$\omega_m$] ++(0, 0);

    \draw (OP1.+)
    to[R=$R_1$] ++(-2, 0)
    to[short, o-, l=$\omega_d$] ++(0, 0);

    \draw (OP1.+)
    to[R=$R_2$, *-] ++(0, -2)
    node[ground] {};

    \draw (OP1.-)
    to[short, *-] ++(0, 1)
    coordinate(t1)
    to[R=$R_2$] (t1 -| OP1.out)
    -- (OP1.out);

    % --- OP2 ---

    \draw (OP1.out)
    to[short, *-, l=$e$] ++(0.5, 0)
    to[R=$R_3$] ++(2, 0)
    node[op amp, anchor=-](OP2) {$OP2$};

    \draw (OP2.-)
    to[short, *-] ++(0, 1)
    coordinate(t2)
    to[R=$R_4$] (t2 -| OP2.out)
    -- (OP2.out);

    \draw (OP2.+)
    node[ground] {};

    \draw (OP2.out)
    to[short, *-] ++(0, 0)
    to[R=$R_7$] ++(0, -2)
    coordinate(t7);

    % --- OP3 ---

    \draw (OP1.out)
    -- ++(0, -2)
    to[potentiometer, n=R5, l=$R_5$] ++(0, -2)
    coordinate(t3);

    \draw (R5.wiper)
    -- (R5.wiper |- t3)
    -- (t3);

    \draw (t3)
    to[short, *-] ++(0, -1)
    to[R=$R_6$] ++(2, 0)
    node[op amp, anchor=-](OP3) {$OP3$};

    \draw (OP3.+)
    node[ground] {};
    
    \draw (OP3.-)
    to[short, *-] ++(0, 1)
    coordinate(t4)
    to[C=$C_1$] (t4 -| OP3.out)
    coordinate(t5)
    -- (OP3.out);

    \draw (t4)
    to[short, *-] ++(0, 1.3)
    coordinate(t6)
    to[open jumper, l=$JP1$] (t6 -| t5)
    to[short, -*] (t5);

    \draw (OP3.out)
    to[short, *-] (OP3.out -| OP2.out)
    to[R, l_=$R_7$] ++(0, 2)
    coordinate (t8)
    to[short, *-*] (t7);

    % --- OP4 ---

    \draw (t8)
    -- ++(1.5, 0)
    node[op amp, anchor=-](OP4) {$OP4$};

    \draw (OP4.+)
    node[ground] {};

    \draw (t7)
    to[R=$R_8$] ++(2, 0)
    coordinate(t9)
    to[potentiometer, n=R9, l=$R_9$] (t9 -| OP4.out)
    coordinate(t10)
    -- (OP4.out);

    \draw (R9.wiper)
    -- (R9.wiper -| t10)
    to[short, -*] (t10);

    \draw (OP4.out)
    to[short, *-o, l=$V_m$] ++(1, 0);
    
\end{circuitikz}
    \caption{PI-regulator krets for hastighetsregulatoren.}
    \label{fig:krets_hastighets_regulator}
\end{figure}

Under er ting fra Øyvind, dette må fjernes senere

\subsection{Tabeller}
Under finner dere kildekoden til Tabell \ref{tab:eksempeltabell}


\begin{table}[tb]
	\centering
	\begin{tabular}{ll} 
	    % ll står for to venstrejusterte kolonner. For høyrejustering kan dere bruke r i stedet. Disse kan selvsagt blandes etter behov. Vi skiller kolonner med &, og avslutter linjer med \\, som dere ser under.
		\toprule
		Variabel & Verdi \\
		\midrule
        $K_p$ & $10.0$ \\
        $K_i$ & $2.0$\\
		\bottomrule
	\end{tabular}
    \caption{Et eksempel på en tabell med regulatorparametre. Merk både at tabellteksten plasseres over tabellen, og at vi vanligvis ikke bruker loddrette linjer.}
\label{tab:eksempeltabell}
\end{table}

\subsection{Sitering og referansehåndtering}
I akademisk skriving er det veldig viktig å sitere kilder. I latex har vi et verktøy kalt BiBTeX som 
håndterer dette. Kilden legges til i en .bib-fil, i vårt tilfelle \texttt{bibliography.bib}. Der kan dere
se hvordan kilder legges inn. For å sitere kilden kan du bruke kommandoen \texttt{\textbackslash{cite}}.
Merk at kilder som ikke siteres, ikke vil dukke opp i kildelisten! Dette betyr at dere i bibliografi-filen kan
legge inn alle kilder dere \emph{tror} dere kommer til å bruke, men de vil ikke dukke opp i kildelisten før du
faktisk siterer dem!

Det finnes mange forskjellige siteringsstiler, og dere kan modifisere mye. Hvis dere vil vite mer om dette
kan dere finne informasjon her \cite{BiberBibtexEtc,WikibookLatex}\footnote{Prøv å begrense sitering av 
nettsider så mye som mulig, og bruk eventuelt \url{http://web.archive.org} hvis dere er bekymret for at en 
referanse kan endre seg i fremtiden.}.

Det finnes mange verktøy for å håndtere referanser. 
De fleste av dere vil nok ikke trenge dem i dette faget, men vi legger ved informasjon, så dere vet om dem 
senere. Noen populære referanseverktøy er JabRef (\url{http://www.jabref.org/}), Mendeley (\url{https://www.mendeley.com/}), EndNote, og Zotero. Alle disse kan eksportere .bib-filer (JabRef lagrer alle
data i en .bib-fil).
