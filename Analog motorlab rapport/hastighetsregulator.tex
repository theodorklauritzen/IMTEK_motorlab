\section{Hastighetsregulator}\label{sec:hastighetsreg}

\subsection{Teori}

Regulatorer brukes for å styre tilstander i en prosess. En regulator regulator får inn et avvik mellom referansen og den målte tilstanden. Regulatorer bruker avviket til å beregne et pådrag til prosessen. P-regulator er en type regulator som lager et pådrag, $u$,  som er proposjonalt med avviket, $e = \omega_d - \omega_m$. $K_p$ er proposjonalitetsleddet i regulatoren, sammenhengen er vist i \autoref{eq:P_regulator}.

\begin{equation}
    \label{eq:P_regulator}
    u(t) = K_p e(t)
\end{equation}

En slik regulator er effektiv så lenge det ikke er en kraft som forhindrer systemet å oppnå en spesifikk tilstand. I et slikt tilfelle vil ikke regulatoren regulere tilstanden til referanseverdien og det til oppstå et stasjonærtavvik. Et integratorledd vil forhindre stasjonærtavvik ved å gi et pådrag som er likt den kraften som forhindrer at systemet når referanse tilstanden. \autoref{eq:PI_regulator} beskriver en PI-regulator,

\begin{equation}
    \label{eq:PI_regulator}
    u(t) = K_p e(t) + \frac{K_p}{T_i} \int_{0}^{t} e(\tau) d\tau
\end{equation}

der $u$ er pådraget, $K_p$ er proposjonalitetskontanten, $e$ er avviket mellom referansen og tilstanden og $T_i$ er integratorkonstanten som representerer tidskonstanten til regulatoren.

\subsection{Metode}

\begin{figure}[b]
    \centering
    \begin{circuitikz} [scale=0.5, transform shape]
    \ctikzset{resistor = european}

    % --- OP1 ---
    \node[op amp](OP1) {$OP1$};
    
    \draw (OP1.-)
    to[R, l_=$R_1$] ++(-2, 0)
    to[short, o-, l=$\omega_m$] ++(0, 0);

    \draw (OP1.+)
    to[R=$R_1$] ++(-2, 0)
    to[short, o-, l=$\omega_d$] ++(0, 0);

    \draw (OP1.+)
    to[R=$R_2$, *-] ++(0, -2)
    node[ground] {};

    \draw (OP1.-)
    to[short, *-] ++(0, 1)
    coordinate(t1)
    to[R=$R_2$] (t1 -| OP1.out)
    -- (OP1.out);

    % --- OP2 ---

    \draw (OP1.out)
    to[short, *-, l=$e$] ++(0.5, 0)
    to[R=$R_3$] ++(2, 0)
    node[op amp, anchor=-](OP2) {$OP2$};

    \draw (OP2.-)
    to[short, *-] ++(0, 1)
    coordinate(t2)
    to[R=$R_4$] (t2 -| OP2.out)
    -- (OP2.out);

    \draw (OP2.+)
    node[ground] {};

    \draw (OP2.out)
    to[short, *-] ++(0, 0)
    to[R=$R_7$] ++(0, -2)
    coordinate(t7);

    % --- OP3 ---

    \draw (OP1.out)
    -- ++(0, -2)
    to[potentiometer, n=R5, l=$R_5$] ++(0, -2)
    coordinate(t3);

    \draw (R5.wiper)
    -- (R5.wiper |- t3)
    -- (t3);

    \draw (t3)
    to[short, *-] ++(0, -1)
    to[R=$R_6$] ++(2, 0)
    node[op amp, anchor=-](OP3) {$OP3$};

    \draw (OP3.+)
    node[ground] {};
    
    \draw (OP3.-)
    to[short, *-] ++(0, 1)
    coordinate(t4)
    to[C=$C_1$] (t4 -| OP3.out)
    coordinate(t5)
    -- (OP3.out);

    \draw (t4)
    to[short, *-] ++(0, 1.3)
    coordinate(t6)
    to[open jumper, l=$JP1$] (t6 -| t5)
    to[short, -*] (t5);

    \draw (OP3.out)
    to[short, *-] (OP3.out -| OP2.out)
    to[R, l_=$R_7$] ++(0, 2)
    coordinate (t8)
    to[short, *-*] (t7);

    % --- OP4 ---

    \draw (t8)
    -- ++(1.5, 0)
    node[op amp, anchor=-](OP4) {$OP4$};

    \draw (OP4.+)
    node[ground] {};

    \draw (t7)
    to[R=$R_8$] ++(2, 0)
    coordinate(t9)
    to[potentiometer, n=R9, l=$R_9$] (t9 -| OP4.out)
    coordinate(t10)
    -- (OP4.out);

    \draw (R9.wiper)
    -- (R9.wiper -| t10)
    to[short, -*] (t10);

    \draw (OP4.out)
    to[short, *-o, l=$V_m$] ++(1, 0);
    
\end{circuitikz}
    \caption{PI-regulator krets for hastighetsregulatoren. Hentet fra \cite{AnalogMotorlabbOppgaver}}
    \label{fig:krets_hastighets_regulator}
\end{figure}

Hastighetsregulatoren ble implementert som en analog PI-regulator som vist i \autoref{fig:krets_hastighets_regulator}. $OP1$ er en differensialforsterker som finner avviket, $e$, transferfunksjonen er gitt ved \autoref{eq:differensialforsterker}.
$OP2$ er en inverterende forsterker som inverterer avviket, uten forsterkning eller demping.
$OP3$ er en integrerende forsterker som integrerer $e$ og forsterker den med $\frac{1}{T_i}$. $JP1$ brukes for å nullstille integratoren og skru av I-leddet i integratoren. Transferfunksjonen til $OP3$ er $-\frac{1}{(R_5 + R_6) C_1} \int e dt$.
$OP4$ summerer spenningen fra $OP2$ og $OP3$ og forsterker resultatet med $K_p$. Transferfunksjonen for $OP4$ er $-\frac{R_8 + R_9}{R_7}(v_2 + v_3)$, der $v_2$ er spenningen ut av $OP2$ og $v_3$ er spenningen ut av $OP3$. Ut fra dette finner vi uttrykk for $K_p$ og $T_i$ som vist i \autoref{eq:K_p_og_T_i}.

\begin{equation}
    \label{eq:K_p_og_T_i}
    K_p = \frac{R_2}{R_1} \frac{R_8 + R_9}{R_7},
    T_i = (R_5 + R_6) C_1
\end{equation}

Størrelsen på motstandene og kondensatoren er vist i \autoref{tab:Komponenter_i_hastighetsregulatoren}

\begin{table}
    \centering
    \caption{Motstander og kondensatorer i hastighetsregulatoren. Verdiene er hentet fra \cite{AnalogMotorlabbOppgaver}}
    \begin{tabular}{lll}
        \toprule
        Størrelse & Verdi & Type \\
		\midrule
        $R_1$, $R_2$ & $100\,k\Omega$ & Resistor\\
        $R_3$, $R_4$, $R_7$, $R_8$ & $10\,k\Omega$ & Resistor \\
        $R_5$, $R_9$ & $1\,M\Omega$ & Potmeter \\
        $R_6$ & $1\,k\Omega$ & Resistor \\
        $C_1$ & $1\,\mu F$ & Kondensator \\
        \bottomrule
    \end{tabular}
    \label{tab:Komponenter_i_hastighetsregulatoren}
\end{table}

\subsection{Resultater}

\begin{figure}[b]
    \centering
    % This file was created with tikzplotlib v0.10.1.
% Dette er eksempel data
\begin{tikzpicture}

\definecolor{darkgray176}{RGB}{176,176,176}
\definecolor{darkorange25512714}{RGB}{255,127,14}
\definecolor{steelblue31119180}{RGB}{31,119,180}

\begin{axis}[
tick align=outside,
tick pos=left,
title={Hastighet PI-regulator},
legend pos=south east,
height=\figH,
width=\figW,
x grid style={darkgray176},
xlabel={\(\displaystyle t\) [ms]},
xmin=-16.1469, xmax=339.0849,
xtick style={color=black},
xtick={-50,0,50,100,150,200,250,300,350},
xticklabels={
  \(\displaystyle {\ensuremath{-}50}\),
  \(\displaystyle {0}\),
  \(\displaystyle {50}\),
  \(\displaystyle {100}\),
  \(\displaystyle {150}\),
  \(\displaystyle {200}\),
  \(\displaystyle {250}\),
  \(\displaystyle {300}\),
  \(\displaystyle {350}\)
},
y grid style={darkgray176},
ylabel={\(\displaystyle V\) [V]},
ymin=-10.410154, ymax=10.751954,
ytick style={color=black},
ytick={-12.5,-10,-7.5,-5,-2.5,0,2.5,5,7.5,10,12.5},
yticklabels={
  \(\displaystyle {\ensuremath{-}12.5}\),
  \(\displaystyle {\ensuremath{-}10.0}\),
  \(\displaystyle {\ensuremath{-}7.5}\),
  \(\displaystyle {\ensuremath{-}5.0}\),
  \(\displaystyle {\ensuremath{-}2.5}\),
  \(\displaystyle {0.0}\),
  \(\displaystyle {2.5}\),
  \(\displaystyle {5.0}\),
  \(\displaystyle {7.5}\),
  \(\displaystyle {10.0}\),
  \(\displaystyle {12.5}\)
}
]
\legend {Respons, Referanse}
\addplot [semithick, steelblue31119180]
table {%
0 -9.2041
0.46199999999974 -9.15527
0.923999999999925 -9.2041
1.38600000000011 -9.30176
1.84799999999985 -9.30176
2.31000000000003 -9.10645
2.77199999999977 -9.25293
3.23399999999996 -9.10645
3.69600000000014 -9.30176
4.15799999999988 -9.39941
4.62000000000007 -9.00879
5.08199999999981 -9.15527
5.54399999999999 -9.30176
6.00600000000018 -9.10645
6.46799999999992 -9.44824
6.9300000000001 -9.25293
7.39199999999984 -8.8623
7.85400000000003 -9.05762
8.31599999999977 -9.10645
8.77799999999995 -9.25293
9.24000000000014 -9.30176
9.70199999999988 -9.00879
10.1640000000001 -9.05762
10.6259999999998 -9.05762
11.088 -9.10645
11.5500000000002 -9.2041
12.0119999999999 -9.39941
12.4740000000001 -9.05762
12.9359999999998 -9.30176
13.398 -9.05762
13.8599999999998 -9.15527
14.3219999999999 -9.30176
14.7840000000001 -9.05762
15.2459999999999 -9.15527
15.7080000000001 -9.30176
16.1699999999998 -8.76465
16.632 -9.00879
17.0940000000002 -9.15527
17.5559999999999 -8.95996
18.0180000000001 -9.15527
18.4799999999998 -9.15527
18.942 -9.15527
19.4039999999998 -9.25293
19.8659999999999 -9.10645
20.3280000000001 -9.2041
20.7899999999999 -9.10645
21.252 -9.05762
21.7139999999998 -9.2041
22.176 -9.2041
22.6380000000002 -9.10645
23.0999999999999 -9.2041
23.5620000000001 -9.05762
24.0239999999998 -9.00879
24.486 -9.10645
24.9479999999997 -9.00879
25.4099999999999 -8.95996
25.8720000000001 -9.05762
26.3339999999999 -8.95996
26.796 -9.15527
27.2579999999998 -9.2041
27.72 -9.10645
28.1820000000002 -9.25293
28.6439999999999 -9.15527
29.1060000000001 -9.15527
29.5679999999998 -9.25293
30.03 -9.30176
30.4919999999997 -9.30176
30.9539999999999 -9.35059
31.4160000000001 -9.25293
31.8779999999999 -9.30176
32.34 -9.30176
32.8019999999998 -9.00879
33.264 -9.25293
33.7260000000001 -9.10645
34.1879999999999 -8.81348
34.6500000000001 -9.05762
35.1119999999998 -9.10645
35.574 -9.15527
36.0360000000002 -9.44824
36.4979999999999 -9.10645
36.9600000000001 -9.25293
37.4219999999998 -9.30176
37.884 -9.15527
38.3459999999998 -9.30176
38.808 -9.2041
39.2700000000001 -9.10645
39.7319999999999 -9.35059
40.1940000000001 -9.30176
40.6559999999998 -9.10645
41.118 -9.25293
41.5800000000002 -9.15527
42.0419999999999 -9.05762
42.5040000000001 -9.30176
42.9659999999998 -9.00879
43.428 -9.05762
43.8899999999998 -9.00879
44.3519999999999 -8.91113
44.8140000000001 -9.15527
45.2759999999999 -9.15527
45.7380000000001 -9.05762
46.1999999999998 -9.25293
46.662 -9.15527
47.1240000000002 -9.25293
47.5859999999999 -9.25293
48.0480000000001 -9.15527
48.5099999999998 -9.15527
48.972 -9.39941
49.4339999999998 -8.95996
49.8959999999999 -9.2041
50.3580000000001 -9.2041
50.8199999999999 -8.95996
51.2820000000001 -9.30176
51.7439999999998 -9.25293
52.206 -8.91113
52.6680000000002 -9.25293
53.1299999999999 -9.00879
53.5920000000001 -9.10645
54.0539999999998 -9.10645
54.516 -9.15527
54.9779999999998 -9.25293
55.4399999999999 -9.35059
55.9020000000001 -8.95996
56.3639999999999 -9.39941
56.826 -9.30176
57.2879999999998 -9.00879
57.75 -9.30176
58.2120000000002 -9.35059
58.6739999999999 -9.30176
59.1360000000001 -9.25293
59.5979999999998 -9.05762
60.06 -9.44824
60.5219999999997 -9.25293
60.9839999999999 -9.00879
61.4460000000001 -9.30176
61.9079999999999 -9.2041
62.37 -8.95996
62.8319999999998 -9.15527
63.294 -9.10645
63.7560000000001 -9.10645
64.2179999999999 -9.15527
64.6800000000001 -9.15527
65.1419999999998 -9.10645
65.604 -9.30176
66.0660000000002 -9.30176
66.5279999999999 -9.35059
66.9900000000001 -9.30176
67.4519999999998 -9.00879
67.914 -9.30176
68.3759999999998 -9.35059
68.838 -8.95996
69.3000000000001 -9.30176
69.7619999999999 -9.15527
70.2240000000001 -8.71582
70.6859999999998 -8.66699
71.148 -8.42285
71.6100000000002 -8.22754
72.0719999999999 -8.3252
72.5340000000001 -7.88574
72.9959999999998 -7.88574
73.458 -7.78809
73.9199999999998 -7.69043
74.3819999999999 -7.25098
74.8440000000001 -7.25098
75.3059999999999 -6.95801
75.7680000000001 -6.90918
76.2299999999998 -6.46973
76.692 -6.61621
77.1540000000002 -6.12793
77.6159999999999 -6.0791
78.0780000000001 -5.88379
78.5399999999998 -5.88379
79.002 -5.83496
79.4639999999998 -5.44434
79.9259999999999 -5.2002
80.3880000000001 -5.15137
80.8499999999999 -5.00488
81.3120000000001 -4.8584
81.7739999999998 -4.66309
82.236 -4.41895
82.6980000000002 -4.12598
83.1599999999999 -3.88184
83.6220000000001 -3.93066
84.0839999999998 -3.73535
84.546 -3.49121
85.0079999999998 -3.44238
85.4699999999999 -2.9541
85.9320000000001 -2.9541
86.3939999999999 -2.85645
86.8560000000001 -2.9541
87.3179999999998 -2.56348
87.78 -2.31934
88.2420000000002 -2.31934
88.7039999999999 -2.0752
89.1660000000001 -1.92871
89.6279999999998 -1.68457
90.09 -1.48926
90.5519999999997 -1.3916
91.0139999999999 -1.0498
91.4760000000001 -0.90332
91.9379999999998 -0.756836
92.4 -0.805664
92.8619999999998 -0.610352
93.324 -0.366211
93.7860000000001 -0.170898
94.2479999999999 -0.170898
94.7100000000001 0.170898
95.1719999999998 0.219727
95.634 0.415039
96.0960000000002 0.561523
96.5579999999999 0.805664
97.0200000000001 0.756836
97.4819999999998 0.90332
97.944 1.24512
98.4059999999998 1.44043
98.868 1.44043
99.3300000000001 1.7334
99.7919999999999 1.97754
100.254 1.97754
100.716 2.17285
101.178 2.31934
101.64 2.6123
102.102 2.66113
102.564 2.80762
103.026 2.85645
103.488 2.80762
103.95 3.14941
104.412 3.2959
104.874 3.39355
105.336 3.54004
105.798 3.6377
106.26 3.88184
106.722 3.88184
107.184 4.12598
107.646 4.12598
108.108 4.37012
108.57 4.56543
109.032 4.56543
109.494 4.5166
109.956 4.80957
110.418 5.00488
110.88 5.00488
111.342 5.24902
111.804 5.24902
112.266 5.2002
112.728 5.68848
113.19 5.49316
113.652 5.63965
114.114 5.54199
114.576 5.78613
115.038 5.98145
115.5 6.03027
115.962 5.93262
116.424 6.12793
116.886 6.27441
117.348 6.27441
117.81 6.32324
118.272 6.4209
118.734 6.56738
119.196 6.66504
119.658 6.4209
120.12 6.71387
120.582 6.71387
121.044 6.7627
121.506 6.66504
121.968 7.10449
122.43 7.20215
122.892 7.10449
123.354 7.15332
123.816 7.25098
124.278 7.44629
124.74 7.10449
125.202 7.34863
125.664 7.49512
126.126 7.2998
126.588 7.39746
127.05 7.49512
127.512 7.54395
127.974 7.54395
128.436 7.69043
128.898 7.88574
129.36 7.69043
129.822 8.03223
130.284 7.73926
130.746 7.78809
131.208 7.78809
131.67 7.93457
132.132 7.93457
132.594 8.12988
133.056 8.12988
133.518 8.08105
133.98 8.12988
134.442 8.12988
134.904 8.22754
135.366 8.22754
135.828 8.3252
136.29 8.37402
136.752 8.3252
137.214 8.22754
137.676 8.42285
138.138 8.37402
138.6 8.52051
139.062 8.47168
139.524 8.27637
139.986 8.52051
140.448 8.52051
140.91 8.37402
141.372 8.66699
141.834 8.61816
142.296 8.22754
142.758 8.61816
143.22 8.76465
143.682 8.66699
144.144 8.91113
144.606 8.56934
145.068 8.66699
145.53 8.8623
145.992 8.76465
146.454 8.71582
146.916 8.61816
147.378 8.81348
147.84 8.81348
148.302 9.00879
148.764 8.8623
149.226 8.76465
149.688 8.91113
150.15 8.71582
150.612 8.8623
151.074 8.91113
151.536 8.81348
151.998 9.2041
152.46 9.15527
152.922 8.8623
153.384 9.10645
153.846 9.10645
154.308 8.95996
154.77 9.35059
155.232 9.10645
155.694 9.10645
156.156 9.10645
156.618 9.05762
157.08 9.10645
157.542 9.15527
158.004 8.95996
158.466 9.25293
158.928 9.15527
159.39 9.05762
159.852 9.15527
160.314 9.15527
160.776 8.95996
161.238 9.49707
161.7 9.30176
162.162 9.25293
162.624 9.30176
163.086 9.00879
163.548 9.30176
164.01 9.2041
164.472 8.95996
164.934 9.35059
165.396 9.30176
165.858 9.2041
166.32 9.39941
166.782 9.2041
167.244 9.2041
167.706 9.30176
168.168 9.15527
168.63 9.30176
169.092 9.30176
169.554 9.10645
170.016 9.35059
170.478 9.44824
170.94 9.25293
171.402 9.39941
171.864 9.15527
172.326 9.49707
172.788 9.35059
173.25 9.35059
173.712 9.35059
174.174 9.44824
174.636 9.2041
175.098 9.35059
175.56 9.39941
176.022 9.00879
176.484 9.39941
176.946 9.2041
177.408 9.15527
177.87 9.35059
178.332 9.15527
178.794 9.59473
179.256 9.44824
179.718 9.35059
180.18 9.59473
180.642 9.35059
181.104 9.44824
181.566 9.44824
182.028 9.30176
182.49 9.64355
182.952 9.49707
183.414 9.39941
183.876 9.59473
184.338 9.44824
184.8 9.49707
185.262 9.49707
185.724 9.39941
186.186 9.49707
186.648 9.30176
187.11 9.35059
187.572 9.5459
188.034 9.49707
188.496 9.30176
188.958 9.64355
189.42 9.30176
189.882 9.39941
190.344 9.64355
190.806 9.44824
191.268 9.64355
191.73 9.49707
192.192 9.30176
192.654 9.59473
193.116 9.35059
193.578 9.35059
194.04 9.5459
194.502 9.30176
194.964 9.49707
195.426 9.39941
195.888 9.35059
196.35 9.59473
196.812 9.39941
197.274 9.49707
197.736 9.79004
198.198 9.49707
198.66 9.74121
199.122 9.64355
199.584 9.39941
200.046 9.59473
200.508 9.49707
200.97 9.44824
201.432 9.5459
201.894 9.30176
202.356 9.39941
202.818 9.49707
203.28 9.44824
203.742 9.64355
204.204 9.39941
204.666 9.44824
205.128 9.59473
205.59 9.59473
206.052 9.74121
206.514 9.64355
206.976 9.44824
207.438 9.64355
207.9 9.49707
208.362 9.49707
208.824 9.39941
209.286 9.2041
209.748 9.10645
210.21 9.15527
210.672 9.05762
211.134 8.95996
211.596 9.05762
212.058 8.76465
212.52 9.00879
212.982 9.10645
213.444 9.05762
213.906 9.2041
214.368 9.15527
214.83 9.10645
215.292 9.15527
215.754 9.30176
216.216 9.2041
216.678 9.10645
217.14 9.00879
217.602 9.15527
218.064 9.10645
218.526 9.00879
218.988 9.30176
219.45 9.15527
219.912 9.00879
220.374 9.2041
220.836 8.91113
221.298 9.25293
221.76 9.25293
222.222 9.05762
222.684 9.39941
223.146 9.35059
223.608 8.95996
224.07 9.25293
224.532 9.10645
224.994 9.15527
225.456 9.10645
225.918 9.15527
226.38 9.05762
226.842 9.00879
227.304 9.25293
227.766 9.10645
228.228 9.25293
228.69 8.81348
229.152 9.10645
229.614 8.95996
230.076 8.8623
230.538 9.05762
231 9.25293
231.462 9.05762
231.924 9.35059
232.386 9.30176
232.848 9.25293
233.31 9.25293
233.772 9.10645
234.234 9.35059
234.696 9.30176
235.158 8.95996
235.62 9.00879
236.082 9.15527
236.544 9.10645
237.006 9.15527
237.468 9.00879
237.93 9.15527
238.392 9.30176
238.854 8.91113
239.316 8.95996
239.778 9.15527
240.24 9.05762
240.702 9.39941
241.164 9.35059
241.626 9.15527
242.088 9.30176
242.55 8.95996
243.012 9.30176
243.474 9.25293
243.936 9.00879
244.398 9.00879
244.86 9.2041
245.322 9.05762
245.784 9.05762
246.246 9.2041
246.708 9.10645
247.17 9.30176
247.632 9.39941
248.094 8.95996
248.556 9.2041
249.018 9.05762
249.48 8.8623
249.942 9.35059
250.404 9.30176
250.866 9.30176
251.328 9.25293
251.79 9.25293
252.252 9.10645
252.714 9.39941
253.176 8.95996
253.638 9.2041
254.1 9.30176
254.562 9.10645
255.024 9.00879
255.486 9.10645
255.948 9.10645
256.41 9.10645
256.872 9.10645
257.334 9.05762
257.796 9.10645
258.258 8.91113
258.72 9.10645
259.182 9.35059
259.644 9.00879
260.106 9.30176
260.568 9.30176
261.03 9.10645
261.492 9.39941
261.954 9.15527
262.416 9.2041
262.878 9.35059
263.34 8.95996
263.802 9.10645
264.264 9.25293
264.726 8.91113
265.188 9.25293
265.65 9.05762
266.112 9.10645
266.574 9.05762
267.036 9.00879
267.498 9.00879
267.96 9.10645
268.422 9.2041
268.884 9.10645
269.346 9.25293
269.808 9.15527
270.27 9.2041
270.732 9.25293
271.194 8.95996
271.656 9.30176
272.118 9.15527
272.58 9.10645
273.042 9.10645
273.504 9.25293
273.966 9.10645
274.428 9.10645
274.89 9.10645
275.352 9.00879
275.814 9.15527
276.276 8.76465
276.738 9.2041
277.2 9.25293
277.662 9.00879
278.124 9.2041
278.586 9.30176
279.048 9.10645
279.51 9.25293
279.972 9.05762
280.434 9.10645
280.896 9.10645
281.358 9.05762
281.82 9.15527
282.282 9.15527
282.744 8.95996
283.206 9.15527
283.668 9.15527
284.13 8.95996
284.592 9.15527
285.054 9.2041
285.516 9.15527
285.978 9.39941
286.44 9.10645
286.902 9.10645
287.364 9.15527
287.826 9.2041
288.288 9.30176
288.75 9.25293
289.212 9.00879
289.674 9.2041
290.136 9.2041
290.598 9.15527
291.06 9.15527
291.522 9.10645
291.984 9.2041
292.446 9.15527
292.908 9.05762
293.37 9.05762
293.832 9.15527
294.294 8.91113
294.756 9.30176
295.218 9.30176
295.68 8.8623
296.142 9.2041
296.604 9.2041
297.066 9.2041
297.528 9.25293
297.99 9.05762
298.452 8.95996
298.914 9.2041
299.376 8.95996
299.838 9.05762
300.3 9.15527
300.762 9.05762
301.224 9.15527
301.686 9.2041
302.148 9.05762
302.61 9.10645
303.072 9.05762
303.534 9.00879
303.996 9.25293
304.458 9.30176
304.92 9.2041
305.382 9.10645
305.844 9.05762
306.306 9.30176
306.768 9.44824
307.23 8.95996
307.692 9.10645
308.154 9.00879
308.616 9.00879
309.078 9.2041
309.54 8.91113
310.002 9.05762
310.464 9.15527
310.926 9.10645
311.388 8.95996
311.85 9.25293
312.312 9.15527
312.774 9.2041
313.236 9.15527
313.698 9.00879
314.16 9.25293
314.622 9.30176
315.084 9.05762
315.546 9.39941
316.008 8.95996
316.47 8.95996
316.932 9.25293
317.394 9.00879
317.856 9.10645
318.318 9.15527
318.78 8.8623
319.242 9.00879
319.704 9.30176
320.166 9.00879
320.628 9.15527
321.09 9.15527
321.552 8.91113
322.014 9.35059
322.476 9.30176
322.938 8.95996
};
\addplot [semithick, darkorange25512714]
table {%
0 -9.05762
0.46199999999974 -8.95996
0.923999999999925 -9.10645
1.38600000000011 -9.15527
1.84799999999985 -9.10645
2.31000000000003 -9.15527
2.77199999999977 -8.95996
3.23399999999996 -9.10645
3.69600000000014 -9.25293
4.15799999999988 -8.95996
4.62000000000007 -9.05762
5.08199999999981 -9.15527
5.54399999999999 -8.95996
6.00600000000018 -9.15527
6.46799999999992 -9.15527
6.9300000000001 -8.95996
7.39199999999984 -9.10645
7.85400000000003 -9.30176
8.31599999999977 -9.15527
8.77799999999995 -9.15527
9.24000000000014 -9.25293
9.70199999999988 -9.15527
10.1640000000001 -8.95996
10.6259999999998 -9.00879
11.088 -8.95996
11.5500000000002 -9.25293
12.0119999999999 -9.10645
12.4740000000001 -9.2041
12.9359999999998 -9.10645
13.398 -9.15527
13.8599999999998 -9.15527
14.3219999999999 -9.2041
14.7840000000001 -9.10645
15.2459999999999 -9.10645
15.7080000000001 -8.95996
16.1699999999998 -9.15527
16.632 -9.10645
17.0940000000002 -9.10645
17.5559999999999 -9.15527
18.0180000000001 -8.95996
18.4799999999998 -9.15527
18.942 -9.00879
19.4039999999998 -9.2041
19.8659999999999 -9.10645
20.3280000000001 -9.2041
20.7899999999999 -8.95996
21.252 -9.05762
21.7139999999998 -9.2041
22.176 -9.10645
22.6380000000002 -9.15527
23.0999999999999 -9.2041
23.5620000000001 -9.05762
24.0239999999998 -9.00879
24.486 -9.2041
24.9479999999997 -9.05762
25.4099999999999 -9.05762
25.8720000000001 -9.10645
26.3339999999999 -9.10645
26.796 -9.10645
27.2579999999998 -9.15527
27.72 -9.25293
28.1820000000002 -9.30176
28.6439999999999 -9.10645
29.1060000000001 -9.00879
29.5679999999998 -9.10645
30.03 -9.10645
30.4919999999997 -9.2041
30.9539999999999 -9.25293
31.4160000000001 -9.05762
31.8779999999999 -9.10645
32.34 -9.05762
32.8019999999998 -9.15527
33.264 -9.25293
33.7260000000001 -9.10645
34.1879999999999 -9.10645
34.6500000000001 -9.25293
35.1119999999998 -9.25293
35.574 -9.10645
36.0360000000002 -9.10645
36.4979999999999 -9.15527
36.9600000000001 -9.10645
37.4219999999998 -9.15527
37.884 -9.10645
38.3459999999998 -9.2041
38.808 -9.10645
39.2700000000001 -9.10645
39.7319999999999 -9.05762
40.1940000000001 -9.2041
40.6559999999998 -9.15527
41.118 -9.15527
41.5800000000002 -9.2041
42.0419999999999 -9.10645
42.5040000000001 -9.10645
42.9659999999998 -9.10645
43.428 -9.15527
43.8899999999998 -9.05762
44.3519999999999 -9.25293
44.8140000000001 -9.10645
45.2759999999999 -9.10645
45.7380000000001 -9.15527
46.1999999999998 -9.25293
46.662 -9.15527
47.1240000000002 -9.10645
47.5859999999999 -9.05762
48.0480000000001 -9.15527
48.5099999999998 -9.00879
48.972 -9.10645
49.4339999999998 -9.15527
49.8959999999999 -8.95996
50.3580000000001 -9.15527
50.8199999999999 -9.10645
51.2820000000001 -9.10645
51.7439999999998 -9.10645
52.206 -9.10645
52.6680000000002 -9.10645
53.1299999999999 -9.05762
53.5920000000001 -9.10645
54.0539999999998 -9.00879
54.516 -9.10645
54.9779999999998 -9.15527
55.4399999999999 -9.10645
55.9020000000001 -9.00879
56.3639999999999 -9.00879
56.826 -9.15527
57.2879999999998 -9.10645
57.75 -9.10645
58.2120000000002 -9.10645
58.6739999999999 -9.2041
59.1360000000001 -9.25293
59.5979999999998 -9.10645
60.06 -9.10645
60.5219999999997 -9.15527
60.9839999999999 -9.10645
61.4460000000001 -9.15527
61.9079999999999 -9.10645
62.37 -9.30176
62.8319999999998 -9.25293
63.294 -9.15527
63.7560000000001 -9.10645
64.2179999999999 -9.10645
64.6800000000001 -9.10645
65.1419999999998 -9.15527
65.604 -9.10645
66.0660000000002 -9.15527
66.5279999999999 -9.15527
66.9900000000001 -9.2041
67.4519999999998 -9.25293
67.914 -9.15527
68.3759999999998 -9.25293
68.838 -9.00879
69.3000000000001 -9.10645
69.7619999999999 9.05762
70.2240000000001 9.10645
70.6859999999998 8.95996
71.148 9.10645
71.6100000000002 9.00879
72.0719999999999 8.95996
72.5340000000001 9.15527
72.9959999999998 9.2041
73.458 9.15527
73.9199999999998 9.10645
74.3819999999999 8.91113
74.8440000000001 9.10645
75.3059999999999 9.15527
75.7680000000001 9.10645
76.2299999999998 9.05762
76.692 9.25293
77.1540000000002 9.10645
77.6159999999999 9.25293
78.0780000000001 9.05762
78.5399999999998 9.10645
79.002 9.25293
79.4639999999998 9.10645
79.9259999999999 9.05762
80.3880000000001 9.00879
80.8499999999999 9.10645
81.3120000000001 9.10645
81.7739999999998 8.95996
82.236 8.95996
82.6980000000002 9.05762
83.1599999999999 9.10645
83.6220000000001 9.05762
84.0839999999998 9.15527
84.546 8.95996
85.0079999999998 9.00879
85.4699999999999 9.2041
85.9320000000001 9.10645
86.3939999999999 8.95996
86.8560000000001 9.15527
87.3179999999998 9.15527
87.78 9.25293
88.2420000000002 9.10645
88.7039999999999 9.10645
89.1660000000001 9.00879
89.6279999999998 9.00879
90.09 9.05762
90.5519999999997 9.10645
91.0139999999999 9.10645
91.4760000000001 9.25293
91.9379999999998 9.05762
92.4 8.95996
92.8619999999998 9.00879
93.324 9.05762
93.7860000000001 9.2041
94.2479999999999 9.10645
94.7100000000001 9.2041
95.1719999999998 9.10645
95.634 9.05762
96.0960000000002 9.05762
96.5579999999999 9.05762
97.0200000000001 8.95996
97.4819999999998 8.95996
97.944 9.10645
98.4059999999998 8.91113
98.868 9.10645
99.3300000000001 9.00879
99.7919999999999 9.00879
100.254 9.00879
100.716 9.10645
101.178 9.05762
101.64 9.15527
102.102 9.05762
102.564 9.05762
103.026 9.10645
103.488 9.10645
103.95 8.91113
104.412 9.05762
104.874 9.05762
105.336 9.15527
105.798 9.10645
106.26 9.10645
106.722 9.10645
107.184 9.10645
107.646 9.00879
108.108 9.05762
108.57 9.10645
109.032 9.10645
109.494 9.2041
109.956 9.05762
110.418 9.10645
110.88 9.25293
111.342 8.95996
111.804 9.15527
112.266 9.15527
112.728 9.10645
113.19 9.00879
113.652 9.25293
114.114 9.15527
114.576 9.05762
115.038 9.10645
115.5 8.95996
115.962 9.15527
116.424 9.10645
116.886 9.2041
117.348 9.00879
117.81 9.10645
118.272 8.95996
118.734 9.00879
119.196 9.10645
119.658 8.81348
120.12 9.15527
120.582 8.95996
121.044 9.05762
121.506 9.00879
121.968 9.10645
122.43 9.10645
122.892 8.95996
123.354 9.10645
123.816 9.10645
124.278 9.15527
124.74 9.05762
125.202 9.10645
125.664 9.10645
126.126 9.05762
126.588 9.05762
127.05 8.95996
127.512 9.10645
127.974 9.10645
128.436 8.8623
128.898 9.10645
129.36 9.10645
129.822 9.10645
130.284 9.10645
130.746 9.10645
131.208 9.10645
131.67 9.05762
132.132 9.10645
132.594 9.05762
133.056 9.10645
133.518 9.05762
133.98 9.10645
134.442 9.00879
134.904 9.00879
135.366 9.10645
135.828 9.10645
136.29 9.00879
136.752 9.00879
137.214 9.15527
137.676 9.2041
138.138 9.10645
138.6 9.10645
139.062 9.2041
139.524 9.05762
139.986 9.10645
140.448 9.10645
140.91 9.00879
141.372 9.2041
141.834 9.00879
142.296 9.10645
142.758 8.95996
143.22 9.10645
143.682 8.95996
144.144 9.10645
144.606 9.05762
145.068 8.95996
145.53 8.95996
145.992 9.00879
146.454 9.10645
146.916 9.05762
147.378 9.00879
147.84 8.95996
148.302 9.05762
148.764 9.10645
149.226 9.10645
149.688 9.05762
150.15 9.00879
150.612 9.10645
151.074 8.95996
151.536 9.00879
151.998 9.10645
152.46 9.05762
152.922 8.95996
153.384 8.95996
153.846 9.10645
154.308 8.95996
154.77 8.95996
155.232 9.10645
155.694 9.05762
156.156 9.05762
156.618 9.15527
157.08 9.15527
157.542 9.15527
158.004 9.05762
158.466 9.10645
158.928 9.15527
159.39 9.05762
159.852 9.10645
160.314 8.95996
160.776 8.81348
161.238 8.95996
161.7 9.10645
162.162 9.25293
162.624 9.15527
163.086 9.15527
163.548 9.15527
164.01 9.05762
164.472 9.00879
164.934 9.00879
165.396 9.05762
165.858 9.10645
166.32 9.10645
166.782 8.95996
167.244 9.25293
167.706 8.91113
168.168 9.05762
168.63 9.00879
169.092 9.15527
169.554 9.05762
170.016 8.95996
170.478 9.15527
170.94 8.95996
171.402 9.00879
171.864 9.10645
172.326 9.25293
172.788 8.95996
173.25 8.95996
173.712 9.15527
174.174 9.15527
174.636 8.95996
175.098 9.10645
175.56 9.05762
176.022 9.10645
176.484 8.95996
176.946 9.00879
177.408 9.2041
177.87 9.15527
178.332 9.05762
178.794 8.81348
179.256 9.05762
179.718 8.91113
180.18 9.00879
180.642 9.00879
181.104 9.15527
181.566 9.10645
182.028 9.05762
182.49 9.10645
182.952 9.10645
183.414 9.15527
183.876 8.95996
184.338 9.15527
184.8 9.05762
185.262 9.05762
185.724 9.15527
186.186 8.95996
186.648 9.00879
187.11 9.05762
187.572 9.15527
188.034 9.05762
188.496 9.05762
188.958 9.10645
189.42 9.10645
189.882 9.00879
190.344 9.10645
190.806 9.10645
191.268 9.10645
191.73 8.95996
192.192 9.10645
192.654 9.15527
193.116 9.10645
193.578 9.00879
194.04 9.10645
194.502 9.10645
194.964 9.15527
195.426 9.10645
195.888 9.00879
196.35 9.25293
196.812 8.95996
197.274 9.10645
197.736 9.25293
198.198 9.10645
198.66 9.25293
199.122 9.05762
199.584 9.2041
200.046 9.10645
200.508 9.05762
200.97 9.05762
201.432 8.95996
201.894 9.10645
202.356 9.00879
202.818 9.05762
203.28 9.00879
203.742 9.10645
204.204 9.10645
204.666 9.05762
205.128 9.10645
205.59 9.00879
206.052 9.10645
206.514 9.05762
206.976 9.25293
207.438 9.10645
207.9 9.00879
208.362 9.00879
208.824 9.10645
209.286 9.05762
209.748 9.00879
210.21 9.2041
210.672 9.05762
211.134 8.95996
211.596 9.00879
212.058 9.15527
212.52 9.10645
212.982 9.00879
213.444 9.25293
213.906 9.15527
214.368 9.10645
214.83 9.00879
215.292 9.05762
215.754 9.10645
216.216 8.91113
216.678 9.10645
217.14 9.15527
217.602 9.15527
218.064 8.8623
218.526 8.95996
218.988 9.05762
219.45 9.00879
219.912 9.15527
220.374 9.05762
220.836 9.00879
221.298 9.10645
221.76 9.10645
222.222 9.00879
222.684 9.10645
223.146 9.05762
223.608 9.05762
224.07 9.05762
224.532 9.30176
224.994 9.2041
225.456 9.25293
225.918 9.05762
226.38 9.05762
226.842 9.10645
227.304 9.2041
227.766 9.00879
228.228 9.10645
228.69 8.91113
229.152 9.10645
229.614 9.10645
230.076 9.05762
230.538 9.10645
231 9.10645
231.462 9.00879
231.924 9.25293
232.386 9.10645
232.848 9.15527
233.31 9.10645
233.772 9.10645
234.234 9.10645
234.696 9.39941
235.158 9.15527
235.62 9.00879
236.082 9.25293
236.544 9.10645
237.006 9.2041
237.468 9.15527
237.93 9.15527
238.392 8.95996
238.854 9.05762
239.316 9.10645
239.778 9.15527
240.24 9.10645
240.702 9.25293
241.164 9.10645
241.626 9.10645
242.088 9.00879
242.55 8.95996
243.012 9.05762
243.474 9.10645
243.936 9.25293
244.398 9.15527
244.86 8.95996
245.322 9.05762
245.784 9.10645
246.246 9.10645
246.708 9.00879
247.17 9.10645
247.632 9.25293
248.094 9.10645
248.556 9.15527
249.018 9.15527
249.48 9.05762
249.942 9.2041
250.404 9.10645
250.866 9.10645
251.328 8.95996
251.79 9.2041
252.252 8.95996
252.714 9.10645
253.176 8.95996
253.638 9.00879
254.1 9.10645
254.562 9.15527
255.024 9.00879
255.486 9.25293
255.948 8.95996
256.41 9.10645
256.872 9.10645
257.334 8.95996
257.796 9.2041
258.258 9.10645
258.72 9.25293
259.182 9.10645
259.644 9.10645
260.106 9.15527
260.568 9.05762
261.03 9.30176
261.492 9.15527
261.954 9.10645
262.416 9.00879
262.878 8.95996
263.34 9.05762
263.802 8.95996
264.264 9.10645
264.726 9.10645
265.188 9.05762
265.65 9.05762
266.112 8.95996
266.574 9.10645
267.036 9.05762
267.498 9.05762
267.96 9.00879
268.422 9.00879
268.884 9.00879
269.346 9.00879
269.808 9.10645
270.27 9.10645
270.732 9.05762
271.194 9.05762
271.656 9.10645
272.118 9.05762
272.58 8.95996
273.042 8.95996
273.504 9.05762
273.966 9.15527
274.428 9.15527
274.89 9.10645
275.352 9.10645
275.814 9.10645
276.276 8.95996
276.738 8.95996
277.2 9.10645
277.662 9.2041
278.124 9.05762
278.586 9.15527
279.048 8.95996
279.51 9.10645
279.972 8.95996
280.434 8.95996
280.896 9.15527
281.358 9.2041
281.82 9.25293
282.282 9.00879
282.744 9.10645
283.206 9.10645
283.668 9.00879
284.13 9.10645
284.592 8.95996
285.054 9.00879
285.516 9.05762
285.978 9.15527
286.44 9.05762
286.902 8.81348
287.364 9.10645
287.826 9.00879
288.288 9.10645
288.75 9.00879
289.212 9.00879
289.674 9.05762
290.136 9.05762
290.598 9.2041
291.06 9.10645
291.522 9.10645
291.984 9.2041
292.446 9.10645
292.908 9.00879
293.37 9.05762
293.832 8.95996
294.294 9.25293
294.756 9.00879
295.218 9.10645
295.68 9.10645
296.142 9.10645
296.604 9.05762
297.066 9.15527
297.528 9.10645
297.99 9.00879
298.452 9.15527
298.914 9.15527
299.376 9.10645
299.838 9.05762
300.3 9.05762
300.762 9.2041
301.224 9.10645
301.686 9.15527
302.148 9.05762
302.61 8.95996
303.072 8.95996
303.534 9.10645
303.996 9.10645
304.458 9.25293
304.92 8.95996
305.382 9.05762
305.844 9.10645
306.306 9.00879
306.768 9.05762
307.23 8.95996
307.692 9.10645
308.154 9.00879
308.616 9.10645
309.078 8.95996
309.54 9.05762
310.002 9.10645
310.464 9.05762
310.926 9.05762
311.388 9.10645
311.85 8.8623
312.312 9.10645
312.774 9.05762
313.236 9.15527
313.698 9.15527
314.16 9.15527
314.622 9.10645
315.084 9.10645
315.546 9.10645
316.008 9.2041
316.47 9.15527
316.932 9.15527
317.394 9.15527
317.856 8.95996
318.318 8.91113
318.78 9.10645
319.242 9.05762
319.704 9.00879
320.166 9.10645
320.628 9.15527
321.09 9.10645
321.552 9.10645
322.014 8.95996
322.476 9.00879
322.938 9.15527
};
\end{axis}

\end{tikzpicture}

    \caption{Sprangresponsen til PI-hastighetsregulatoren. Den oransje grafen viser referansen og den blå grafen viser responsen til PI-regulatoren Data er hentet fra \cite{EksempelData}}
    \label{fig:hastighet_PI_regulator}
\end{figure}

\subsection{Diskusjon}
