\section{Hastighetsregulator}\label{sec:hastighetsreg}

\subsection{Teori}

Regulatorer brukes for å styre tilstander i en prosess. En regulator regulator får inn et avvik mellom referansen og den målte tilstanden. Regulatorer bruker avviket til å beregne et pådrag til prosessen. P-regulator er en type regulator som lager et pådrag, $u$,  som er proposjonalt med avviket, $e = \omega_d - \omega_m$. $K_p$ er proposjonalitetsleddet i regulatoren, sammenhengen er vist i \autoref{eq:P_regulator}.

\begin{equation}
    \label{eq:P_regulator}
    u(t) = K_p e(t)
\end{equation}

En slik regulator er effektiv så lenge det ikke er en kraft som forhindrer systemet å oppnå en spesifikk tilstand. I et slikt tilfelle vil ikke regulatoren regulere tilstanden til referanseverdien og det til oppstå et stasjonærtavvik. Et integratorledd vil forhindre stasjonærtavvik ved å gi et pådrag som er likt den kraften som forhindrer at systemet når referanse tilstanden. \autoref{eq:PI_regulator} beskriver en PI-regulator,

\begin{equation}
    \label{eq:PI_regulator}
    u(t) = K_p e(t) + \frac{K_p}{T_i} \int_{0}^{t} e(\tau) d\tau
\end{equation}

der $u$ er pådraget, $K_p$ er proposjonalitetskontanten, $e$ er avviket mellom referansen og tilstanden og $T_i$ er integratorkonstanten som representerer tidskonstanten til regulatoren.

\subsection{Metode}

\begin{figure}[b]
    \centering
    \begin{circuitikz} [scale=0.5, transform shape]
    \ctikzset{resistor = european}

    % --- OP1 ---
    \node[op amp](OP1) {$OP1$};
    
    \draw (OP1.-)
    to[R, l_=$R_1$] ++(-2, 0)
    to[short, o-, l=$\omega_m$] ++(0, 0);

    \draw (OP1.+)
    to[R=$R_1$] ++(-2, 0)
    to[short, o-, l=$\omega_d$] ++(0, 0);

    \draw (OP1.+)
    to[R=$R_2$, *-] ++(0, -2)
    node[ground] {};

    \draw (OP1.-)
    to[short, *-] ++(0, 1)
    coordinate(t1)
    to[R=$R_2$] (t1 -| OP1.out)
    -- (OP1.out);

    % --- OP2 ---

    \draw (OP1.out)
    to[short, *-, l=$e$] ++(0.5, 0)
    to[R=$R_3$] ++(2, 0)
    node[op amp, anchor=-](OP2) {$OP2$};

    \draw (OP2.-)
    to[short, *-] ++(0, 1)
    coordinate(t2)
    to[R=$R_4$] (t2 -| OP2.out)
    -- (OP2.out);

    \draw (OP2.+)
    node[ground] {};

    \draw (OP2.out)
    to[short, *-] ++(0, 0)
    to[R=$R_7$] ++(0, -2)
    coordinate(t7);

    % --- OP3 ---

    \draw (OP1.out)
    -- ++(0, -2)
    to[potentiometer, n=R5, l=$R_5$] ++(0, -2)
    coordinate(t3);

    \draw (R5.wiper)
    -- (R5.wiper |- t3)
    -- (t3);

    \draw (t3)
    to[short, *-] ++(0, -1)
    to[R=$R_6$] ++(2, 0)
    node[op amp, anchor=-](OP3) {$OP3$};

    \draw (OP3.+)
    node[ground] {};
    
    \draw (OP3.-)
    to[short, *-] ++(0, 1)
    coordinate(t4)
    to[C=$C_1$] (t4 -| OP3.out)
    coordinate(t5)
    -- (OP3.out);

    \draw (t4)
    to[short, *-] ++(0, 1.3)
    coordinate(t6)
    to[open jumper, l=$JP1$] (t6 -| t5)
    to[short, -*] (t5);

    \draw (OP3.out)
    to[short, *-] (OP3.out -| OP2.out)
    to[R, l_=$R_7$] ++(0, 2)
    coordinate (t8)
    to[short, *-*] (t7);

    % --- OP4 ---

    \draw (t8)
    -- ++(1.5, 0)
    node[op amp, anchor=-](OP4) {$OP4$};

    \draw (OP4.+)
    node[ground] {};

    \draw (t7)
    to[R=$R_8$] ++(2, 0)
    coordinate(t9)
    to[potentiometer, n=R9, l=$R_9$] (t9 -| OP4.out)
    coordinate(t10)
    -- (OP4.out);

    \draw (R9.wiper)
    -- (R9.wiper -| t10)
    to[short, -*] (t10);

    \draw (OP4.out)
    to[short, *-o, l=$V_m$] ++(1, 0);
    
\end{circuitikz}
    \caption{PI-regulator krets for hastighetsregulatoren. Hentet fra \cite{AnalogMotorlabbOppgaver}}
    \label{fig:krets_hastighets_regulator}
\end{figure}

Hastighetsregulatoren ble implementer som en analog PI-regulator som vist i \autoref{fig:krets_hastighets_regulator}. $OP1$ er en differensialforsterker som finner avviket, $e$, transferfunksjonen er gitt ved \autoref{eq:differensialforsterker}.
$OP2$ er en inverterende forsterker som inverterer avviket, uten forsterkning eller demping.
$OP3$ er en integrerende forsterker som integrerer $e$ og forsterker den med $\frac{1}{T_i}$. $JP1$ brukes for å nullstille integratoren og skru av I-leddet i integratoren. Transferfunksjonen til $OP3$ er $-\frac{1}{(R_5 + R_6) C_1} \int e dt$.
$OP4$ summerer spenningen fra $OP2$ og $OP3$ og forsterker resultatet med $K_p$. Transferfunksjonen for $OP4$ er $-\frac{R_8 + R_9}{R_7}(v_2 + v_3)$, der $v_2$ er spenningen ut av $OP2$ og $v_3$ er spenningen ut av $OP3$. Ut fra dette finner vi uttrykk for $K_p$ og $T_i$ som vist i \autoref{eq:K_p_og_T_i}.

\begin{equation}
    \label{eq:K_p_og_T_i}
    K_p = \frac{R_2}{R_1} \frac{R_8 + R_9}{R_7},
    T_i = (R_5 + R_6) C_1
\end{equation}

Størrelsen på motstandene og kondensatoren er vist i \autoref{tab:Komponenter_i_hastighetsregulatoren}

\begin{table}
    \centering
    \caption{Motstander og kondensatorer i hastighetsregulatoren. Verdiene er hentet fra \cite{AnalogMotorlabbOppgaver}}
    \begin{tabular}{lll}
        \toprule
        Størrelse & Verdi & Type \\
		\midrule
        $R_1$, $R_2$ & $100\,k\Omega$ & Resistor\\
        $R_3$, $R_4$, $R_7$, $R_8$ & $10\,k\Omega$ & Resistor \\
        $R_5$, $R_9$ & $1\,M\Omega$ & Potmeter \\
        $R_6$ & $1\,k\Omega$ & Resistor \\
        $C_1$ & $1\,\mu F$ & Kondensator \\
        \bottomrule
    \end{tabular}
    \label{tab:Komponenter_i_hastighetsregulatoren}
\end{table}

\subsection{Resultater}

\subsection{Diskusjon}
