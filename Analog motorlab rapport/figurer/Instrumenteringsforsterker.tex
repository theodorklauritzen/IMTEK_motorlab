\begin{circuitikz} [scale=0.6, transform shape]
\ctikzset{resistor = european}

\node [op amp, noinv input up] (OP){$OP1$};

\draw (OP.+)
    -- ++(-2,0)
    to[short,o-,l=$V_1$] ++(0,0);

\draw (OP.-)
    -- ++(-1,0)
    to[short,-*] ++(0, -1)
    coordinate(o1)
    to[R,l_=$R_1$] (o1 -| OP.out)
    -- (OP.out)
    to[short,*-] ++(0.5,0)
    -- ++(0,-1)
    to[R,l=$R_2$] ++(2,0)
    coordinate(x2)
    to[short,*-] (x2)
    -- ++(0,-1)
    -- ++(1,0)
    node[op amp, anchor=-](OP2){$OP3$};

\draw (x2)
    to[R,l=$R_3$] (x2 -| OP2.out)
    -- (OP2.out)
    to[short,*-] ++(1,0)
    to[short,o-,l_=$v_{out}$] ++(0,0);

\draw (o1)
    -- ++(-0.5,0)
    to[potentiometer,l_=$R_G$] ++(0,-2)
    -- ++(0.5,0)
    coordinate(o2)
    to[short,*-] (o2)
    -- ++(0,-1)
    -- ++(1,0)
    node[op amp, anchor=-](OP3){$OP2$};
    
\draw (OP3.+)
    -- ++(-2,0)
    to[short,o-,l=$V_2$] ++(0,0);

\draw (o2)
    to[R,l=$R_1$] (o2 -| OP3.out)
    -- (OP3.out)
    to[short,*-] ++(0.5,0)
    coordinate(temp2)
    -- (OP2.+ -| temp2)
    to[R,l=$R_2$] ++(2,0)
    coordinate(x3)
    to[short,*-] (OP2.+)
    (x3)
    -- ++(0,-0.5)
    to[R,l=$R_3$] ++(0,-2)
    node[ground] {}++(0,0);
    
    
    
    

\end{circuitikz}