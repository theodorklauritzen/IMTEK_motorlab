
%[] fikse referanser
%[] sette inn likninger



\section{Hastighetsmåler}
\subsection{Teori}
Differensialforsterkeren i \autoref{fig:differensialforsterker} er er operasjonsforsterkerkrets som forsterker opp forskjellen på de to signalene/spenningene som blir satt inn på kretsen. En differensialforsterker av typen i \autoref{fig:differensialforsterker} vil med en ideell operasjonsforsterker kunne utrykkes ved transferfunksjonen\cite{Johnson}:
\begin{equation}
    V_{out} = \frac{R_2}{R_1}(V_2-V_1)
    \label{eq:differensialforsterker}
\end{equation}

En Instrumenteringsforsterker som vist i \autoref{fig:instrumenteringsforsterker} er en mer komplisert versjon av en differensialforsterker, men som har noen viktige fordeler, hver av inngangene i instrumenteringsforsterkeren blir blant annet buffret slik at inngangsimpedansen til kretsen blir høy. Dette fungerer fordi det målte signalet blir sendt rett inn i portene på opampene der inngangsimpedansen er høy, og signalet har ingen andre veier å gå sånn som det har i differensialforsterkeren. Forsterkningen i kretsen kan også endres ved å endre på $R_G$. Instrumenteringsforsterkeren i \autoref{fig:instrumenteringsforsterker} kan dersom man antar en ideell operasjonsforsterker uttrykkes ved transferfunksjonen:
\begin{equation}
    V_{out} = (1+\frac{2R_1}{R_G})(\frac{R_3}{R_2})(V_2-V_1)
    \label{eq:instrumenteringsforsterker}
\end{equation}


\begin{figure} [h]
    
\begin{circuitikz} [scale=0.8,transform shape]
\ctikzset{resistor = european}

\node[op amp](OP){$OP1$};

\draw (OP.-)
    -- ++(-0.5,0)
    coordinate(N1)
    to[short,*-] ++(-0.5,0)
    to[R,l_=$R_1$] ++(-2,0)
    -- ++(-0.5,0)
    to[short,o-,l=$v_1$] ++(0,0);

\draw (OP.+)
    -- ++(-0.5,0)
    coordinate(N2)
    to[short,*-] ++(-0.5,0)
    to[R=$R_1$] ++(-2,0)
    -- ++(-0.5,0)
    to[short,o-,l=$v_2$] ++(0,0);
    
\draw (N1)
    -- ++(0,1)
    coordinate(temp1)
    to[R=$R_2$] (temp1 -| OP.out)
    -- (OP.out)
    to[short,*-] ++(1,0)
    to[short,o-,l_=$v_{out}$] ++(0,0);

\draw (N2)
    -- ++(0,-0.5)
    to[R=$R_2$] ++(0,-2)
    node[ground]{} ++(0,0);

\end{circuitikz}

    \caption{Differensiell operasjonsforsterer. Figur hentet fra \cite{Johnson}}
    \label{fig:differensialforsterker}
\end{figure}


\begin{figure} [h]
    \centering
    \begin{circuitikz} [scale=0.6, transform shape]
\ctikzset{resistor = european}

\node [op amp] (OP){};

\draw (OP.-)
    -- ++(-2,0)
    to[short,o-,l=$V_1$] ++(0,0);

\draw (OP.+)
    -- ++(-1,0)
    to[short,-*] ++(0, -1)
    coordinate(o1)
    to[R,l_=$R_1$] (o1 -| OP.out)
    -- (OP.out)
    to[short,*-] ++(0.5,0)
    -- ++(0,-1)
    to[R,l=$R_2$] ++(2,0)
    coordinate(x2)
    to[short,*-] (x2)
    -- ++(0,-1)
    -- ++(1,0)
    node[op amp, anchor=-](OP2){};

\draw (x2)
    to[R,l=$R_3$] (x2 -| OP2.out)
    -- (OP2.out)
    to[short,*-] ++(1,0)
    to[short,o-,l_=$v_{out}$] ++(0,0);

\draw (o1)
    -- ++(-0.5,0)
    to[potentiometer,l_=$R_G$] ++(0,-2)
    -- ++(0.5,0)
    coordinate(o2)
    to[short,*-] (o2)
    -- ++(0,-1)
    -- ++(1,0)
    node[op amp, anchor=-](OP3){};
    
\draw (OP3.+)
    -- ++(-2,0)
    to[short,o-,l=$V_2$] ++(0,0);

\draw (o2)
    to[R,l=$R_1$] (o2 -| OP3.out)
    -- (OP3.out)
    to[short,*-] ++(0.5,0)
    coordinate(temp2)
    -- (OP2.+ -| temp2)
    to[R,l=$R_2$] ++(2,0)
    coordinate(x3)
    to[short,*-] (OP2.+)
    (x3)
    -- ++(0,-0.5)
    to[R,l=$R_3$] ++(0,-2)
    node[ground] {}++(0,0);
    
    
    
    

\end{circuitikz}
    \caption{Intrumenteringsforsterker. Figur hentet fra \cite{Johnson}}
    \label{fig:instrumenteringsforsterker}
\end{figure}



\subsection{Metode}

For å måle hastigheten, var et tachometer montert på den ene enden av motoren. Dette tachometeret ga ut hastigheten til motoren i form av en differensiell spenning i intervallet $\pm1.79\,V$. For å oppnå ønsket spenningsnivå i $\omega_m$ på $\pm10\,V$ ble en differensiellforsterker (\autoref{fig:differensialforsterker}) og en instrumenteringsforsterker (\autoref{fig:instrumenteringsforsterker}) implementert på studentkortet.
Hensikten med å implementere både differensiellforsterkeren og instrumenteringsforsterkeren var å sammenlikne de to spesielt med tanke på variasjon i utgangsimpedanse fra tachometeret.




-----------------------------

Måling av motorhastighet ble implementert gjennom bruken av to forskjellige differensielle forsterkere, en 

hvordan hver av de brukes

Forsterkning og resistorverdier

Referere til figurer i tekst



\begin{table}[h]
	\centering
    \caption{Motstander i differensiellforsterkeren}
	\begin{tabular}{lll}
		\toprule
		Størrelse & Verdi & Type \\
		\midrule
        $R_1$ & $10\,k\Omega$& Resistor \\
        $R_2$ & $56\,k\Omega$ & Resistor\\
		\bottomrule
	\end{tabular}
\label{tab:eksempeltabell}
\end{table}


\begin{table}[h]
	\centering
    \caption{Motstander i Instrumenteringsforsterkeren}
	\begin{tabular}{lll}
		\toprule
		Størrelse & Verdi & Type \\
		\midrule
        $R_1, R_3$ & $10\,k\Omega$& Resistor \\
        $R_2$ & $18\,k\Omega$ & Resistor\\
        $R_G$ & $10\,k\Omega$ & Potmeter\\
		\bottomrule
	\end{tabular}
    
\label{tab:eksempeltabell}
\end{table}

\subsection{Resultater}
Vise til grafer

\textbf{insert grafer her}



\subsection{Diskusjon}

Forventet resultat

Eventuelt hva annet vi forventet

Som vi ser i  \autoref{fig:differensialforsterker}


------------------------------------------

% For hver del bør dere gå gjennom hva dere skal gjøre, hvordan dere gjør det, resultatene dere fikk,
% og diskusjon av disse. 

% Husk å referere til alle figurer i teksten. Dersom dere ikke referer til en 
% figur i teksten indikerer det enten at figuren ikke er relevant til det dere
% har skrevet (fjern figuren), eller at teksten deres ikke er komplett (skriv om).
% Hver figur har et nummer som dere kan referere til. Dette kan \LaTeX gjøre for dere
% på følgende måte. 



% Hver figur har en \texttt{label}. Denne etiketten kan dere
% referere til ved å bruke kommandoen \texttt{ref} (Det samme gjelder også for tabeller).
% Figurer (og tabeller) vil generelt ikke dukke opp der du har plassert den i teksten.
% Dette er helt greit, og gir typisk en penere rapport enn om man tvinger figurer til å dukke opp på et bestemt sted. I kildekoden for figurer vil dere se at det finnes et
% valgfritt argument \verb+[htb]+. Dette lar dere spesifisere hvor figuren bør plasseres. Her står \verb+h+ for her, \verb+t+ står for toppen av siden, og \verb+b+
% står for bunnen av siden. Generelt blir det penere med figurer og tabeller som legges på toppen eller bunnen av en side. Det finnes også en mulighet \verb+p+ for å legge figuren på en side sammen med andre figurer og tabeller. (Dere kan legge inn flere valg, i prioritert rekkefølge). Merk at figuren ikke alltid blir plassert slik du har bedt om. Dette er fordi LaTeX optimaliserer hvordan alt plasseres. Det finnes muligheter for å tvinge plassering, men dette er sterkt frarådet ettersom det ofte fører til rar layout et eller annet sted.




% \begin{figure}[b]
% 	\centering
% 	\includegraphics[width=0.40\textwidth]{figurer/itk_ntnu.jpg}
% 	\caption{En figur med logoen til ITK.}
% \label{fig:layers_openloop}
% \end{figure}