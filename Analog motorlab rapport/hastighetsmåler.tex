\section{Hastighetsmåler}
\subsection{INTRODUSER nødvendig teori}
Håntering og justering av differensielle signaler

Hastighetsregulator for å få motoren til å følge en ønsket hastighet

To typer differensialforsterkere






\subsection{Beskriv implementasjon}
Tegning av begge kretsene

hvordan hver av de brukes

Forsterkning og resistorverdier

Referere til figurer i tekst



\begin{figure} [h]
    \centering
    \begin{circuitikz} [scale=0.6, transform shape]
\ctikzset{resistor = european}

\node [op amp] (OP){};

\draw (OP.-)
    -- ++(-2,0)
    to[short,o-,l=$V_1$] ++(0,0);

\draw (OP.+)
    -- ++(-1,0)
    to[short,-*] ++(0, -1)
    coordinate(o1)
    to[R,l_=$R_1$] (o1 -| OP.out)
    -- (OP.out)
    to[short,*-] ++(0.5,0)
    -- ++(0,-1)
    to[R,l=$R_2$] ++(2,0)
    coordinate(x2)
    to[short,*-] (x2)
    -- ++(0,-1)
    -- ++(1,0)
    node[op amp, anchor=-](OP2){};

\draw (x2)
    to[R,l=$R_3$] (x2 -| OP2.out)
    -- (OP2.out)
    to[short,*-] ++(1,0)
    to[short,o-,l_=$v_{out}$] ++(0,0);

\draw (o1)
    -- ++(-0.5,0)
    to[potentiometer,l_=$R_G$] ++(0,-2)
    -- ++(0.5,0)
    coordinate(o2)
    to[short,*-] (o2)
    -- ++(0,-1)
    -- ++(1,0)
    node[op amp, anchor=-](OP3){};
    
\draw (OP3.+)
    -- ++(-2,0)
    to[short,o-,l=$V_2$] ++(0,0);

\draw (o2)
    to[R,l=$R_1$] (o2 -| OP3.out)
    -- (OP3.out)
    to[short,*-] ++(0.5,0)
    coordinate(temp2)
    -- (OP2.+ -| temp2)
    to[R,l=$R_2$] ++(2,0)
    coordinate(x3)
    to[short,*-] (OP2.+)
    (x3)
    -- ++(0,-0.5)
    to[R,l=$R_3$] ++(0,-2)
    node[ground] {}++(0,0);
    
    
    
    

\end{circuitikz}
    \caption{Intrumenteringsforsterker HUSK Å REFERERE TIL KILDE}
    \label{fig:instrumenteringsforsterker}
\end{figure}

\begin{figure} [h]
    
\begin{circuitikz} [scale=0.8,transform shape]
\ctikzset{resistor = european}

\node[op amp](OP){$OP1$};

\draw (OP.-)
    -- ++(-0.5,0)
    coordinate(N1)
    to[short,*-] ++(-0.5,0)
    to[R,l_=$R_1$] ++(-2,0)
    -- ++(-0.5,0)
    to[short,o-,l=$v_1$] ++(0,0);

\draw (OP.+)
    -- ++(-0.5,0)
    coordinate(N2)
    to[short,*-] ++(-0.5,0)
    to[R=$R_1$] ++(-2,0)
    -- ++(-0.5,0)
    to[short,o-,l=$v_2$] ++(0,0);
    
\draw (N1)
    -- ++(0,1)
    coordinate(temp1)
    to[R=$R_2$] (temp1 -| OP.out)
    -- (OP.out)
    to[short,*-] ++(1,0)
    to[short,o-,l_=$v_{out}$] ++(0,0);

\draw (N2)
    -- ++(0,-0.5)
    to[R=$R_2$] ++(0,-2)
    node[ground]{} ++(0,0);

\end{circuitikz}

    \caption{Differensiell operasjonsforsterer HUSK Å REFERERE}
    \label{fig:differensialforsterker}
\end{figure}


\subsection{Presenter resultater}
Vise til grafer

\textbf{insert graf her}



\subsection{Diskuter resultater}

Forventet resultat

Eventuelt hva annet vi forventet

Som vi ser i \texttt{Figur} \ref{fig:differensialforsterker}


------------------------------------------


For hver del bør dere gå gjennom hva dere skal gjøre, hvordan dere gjør det, resultatene dere fikk,
og diskusjon av disse. 

Husk å referere til alle figurer i teksten. Dersom dere ikke referer til en 
figur i teksten indikerer det enten at figuren ikke er relevant til det dere
har skrevet (fjern figuren), eller at teksten deres ikke er komplett (skriv om).
Hver figur har et nummer som dere kan referere til. Dette kan \LaTeX gjøre for dere
på følgende måte. 



Hver figur har en \texttt{label}. Denne etiketten kan dere
referere til ved å bruke kommandoen \texttt{ref} (Det samme gjelder også for tabeller).
Figurer (og tabeller) vil generelt ikke dukke opp der du har plassert den i teksten.
Dette er helt greit, og gir typisk en penere rapport enn om man tvinger figurer til å dukke opp på et bestemt sted. I kildekoden for figurer vil dere se at det finnes et
valgfritt argument \verb+[htb]+. Dette lar dere spesifisere hvor figuren bør plasseres. Her står \verb+h+ for her, \verb+t+ står for toppen av siden, og \verb+b+
står for bunnen av siden. Generelt blir det penere med figurer og tabeller som legges på toppen eller bunnen av en side. Det finnes også en mulighet \verb+p+ for å legge figuren på en side sammen med andre figurer og tabeller. (Dere kan legge inn flere valg, i prioritert rekkefølge). Merk at figuren ikke alltid blir plassert slik du har bedt om. Dette er fordi LaTeX optimaliserer hvordan alt plasseres. Det finnes muligheter for å tvinge plassering, men dette er sterkt frarådet ettersom det ofte fører til rar layout et eller annet sted.




\begin{figure}[b]
	\centering
	\includegraphics[width=0.40\textwidth]{figurer/itk_ntnu.jpg}
	\caption{En figur med logoen til ITK.}
\label{fig:layers_openloop}
\end{figure}
