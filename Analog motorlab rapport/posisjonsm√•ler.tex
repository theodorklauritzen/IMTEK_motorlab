\section{Posisjonsmåler}
\label{sec:pos_måler}

\subsection{Teori}

% Enpolet målesignal
%   - Referanse til signaljord
% Signalkilde -[signal]-> signalmottaker / last
% Målevariabel -> Føler -> Omsetter -> Målesignal

% Standardiserte signalnivåer
% 1 - 5V
% Elevated Zero
%   - Muliggjør deteksjon av feil
%   - Vanskelig å måle 0, krever negativ strømforsyning
%   - Ledningsbrud
%   - Muliggjør alarmgrenser utenfor ordinært måleområde

Posisjonsmåling kan enten være en vinkelmåling eller en strekningsmåling. For en servomotor er målevariabelen vinkel nyttigst. Potensiometer kan brukes som en føler for måleveraibelen. Med en spennig over potensiometeret, vil dette generere et målesignal.

For å lettere detektere feil er standardiserte målesignaler med elevated zero nyttig. Ledningsbrudd blir da lettere å oppdage, i tillegg er det vanskelig å måle spenninger rundt $0V$, med laveste signal over $0V$ vil ikke dette bli et problem. Derfor er det lurt å omformere målesignalet til et standardisert målesignal.

\subsection{Metode}

\begin{figure}[h]
    \centering
    \begin{circuitikz}[scale=0.7, transform shape]
    \ctikzset{resistor = european}

    \draw (0, 0)
    to[short, o-, l=15<\volt>] ++(0, 0)
    to[R=12<\kilo\ohm>] ++(2, 0)
    to[potentiometer, n=R2] ++(0, -2)
    to[R=8.2<\kilo\ohm>] ++(-2, 0)
    to[short, -o, l=-15<\volt>] ++(0, 0);

    \draw[->] (R2) ++(-0.4, 0.3)
    -- ++(0, -0.6);
    
    \draw (R2) ++(0.2, 0.4)
    node[anchor=west]{\SI{5}{\kilo\ohm}};

    \draw (R2.wiper)
    to[short, -o] ++(0.5, 0)
    node[anchor=west]{$\theta$};
    
\end{circuitikz}
    \caption{Potensiometeret i motorkortet. Figur hentet fra \cite{AnalogMotorlabbOppgaver}}
    \label{fig:posisjon_maler_potmeter}
\end{figure}

I motorkortet er et potensiometer koblet til motorakslingen. Den er koblet i serie med to motstander, som vist i \autoref{fig:posisjon_maler_potmeter}. Spenningsdeling brukes for å kommefram til følgende uttryk for $V(\theta)$

\begin{equation}
    \label{eq:V_av_theta}
    V(\theta) = \frac{\SI{5}{\kilo\ohm} + R(\theta)}{\SI{12}{\kilo\ohm} + \SI{5}{\kilo\ohm} + \SI{8.2}{\kilo\ohm}} \SI{30}{\volt} - \SI{15}{\volt},
\end{equation}

der $V(\theta)$ er spenningen ut fra motorkortet, $R(\theta)$ er motstaden i potensiometeret gitt vinkelen til motoren. Det betyr at $V(\theta_{min}) = \SI{0.71428}{\volt}$ og $V(\theta_{max}) =\SI{-5.23809}{\volt}$. 

\begin{figure}[h]
    \centering
    \begin{circuitikz} [scale=0.6, transform shape]
    \ctikzset{resistor = european}

    % --- OP1 ---
    \node[op amp](OP1) {$OP1$};
    
    \draw (OP1.+)
    -- ++(-0.3, 0)
    to[short, o-, l=$V(\theta)$] ++(0, 0);

    \draw (OP1.-)
    -- ++(0, 1)
    coordinate(t1)
    -- (t1 -| OP1.out)
    to[short, -*] (OP1.out);

    % --- OP2 ---
    \draw (OP1.out)
    to[R=$R_1$] ++(2, 0)
    node[op amp, anchor=-](OP2) {$OP2$};

    \draw (OP2.+)
    node[ground]{};

    \draw (OP2.-)
    to[short, *-] ++(0, 1)
    coordinate(t2)
    to[R=$R_1$] (t2 -| OP2.out)
    to[short, -*] (OP2.out);

    % --- OP3 ---
    \draw (OP2.out)
    to[R=$R_3$] ++(2, 0)
    node[op amp, anchor=-](OP3) {$OP3$};

    \draw (OP3.+)
    ++(-1, 0)
    to[short, o-, l=$+15V$] ++(0, 0)
    to[potentiometer, l_=$R_3$, n=R3] ++(0, -2)
    to[short, -o, l=$-15V$] ++(0, 0);

    \draw (OP3.+)
    -- (OP3.+ |- R3.wiper)
    -- (R3.wiper);

    \draw (OP3.-)
    -- ++(0, 1)
    coordinate(t3)
    to[potentiometer, l=$R_4$, n=R4] (t3 -| OP3.out)
    -- (OP3.out);

    \draw (t3)
    -- (t3 |- R4.wiper)
    -- (R4.wiper);

    \draw (OP3.out)
    to[short, *-o, l_=$\theta_m$] ++(0.3, 0);
    
\end{circuitikz}
    \caption{Krets for å omforme spenningen fra posisjonsmåleren til et $1\,V$ til $5\,V$ signal. Figur hentet fra \cite{AnalogMotorlabbOppgaver}}
    \label{fig:krets_posisjons_maler}
\end{figure}

Transferfunksjonen i \autoref{fig:krets_posisjons_maler} kan deles opp i to deler. De to første operasjonsforsterkerne har denne transferfuksjonen $V_x = -V(\theta)$. Den neste bruker vi opampens gyldne regler for å sette opp følgende uttrykk
\begin{equation}
    \label{eq:posisjonmåler_skalering}
    (V_x - V_y) \frac{1}{R_2} = (V_y - V_{\theta_m}(\theta)) \frac{1}{R_4},
\end{equation}
der $V_x$, $V_y$ og $V_{\theta_m}$ er potensialet i punktene navngitt i \autoref{fig:krets_posisjons_maler} og $R_2$ og $R_4$ er motstandene i same figur.

For å finne $R_3$ og $R_4$ må vi løse for $V_y$ og$R_4$. Da bruker vi sammenhengene $V_{\theta_m}(\theta_{min}) = \SI{1}{\volt}$ og $V_{\theta_m}(\theta_{max}) = \SI{5}{\volt}$. Da blir $R_4 = \SI{4.6}{\kilo\ohm}$ og 
$V_y = \SI{2.7}{\volt}$. Sammenhengen mellom $R_3$ og $V_y$ er gitt ved $\frac{R_3}{\SI{100}{\kilo\ohm}} \SI{30}{\volt} = V_y + \SI{15}{\volt}$, da blir $R_3 = \SI{59}{\kilo\ohm}$.

\begin{table}[h]
    \centering
    \caption{Motstander og kondensatorer i posisjonomformer. Verdiene er hentet fra \cite{AnalogMotorlabbOppgaver}}
    \begin{tabular}{lll}
        \toprule
        Størrelse & Verdi & Type \\
		\midrule
        $R_1$ & $1\,k\Omega$ & Resistor\\
        $R_2$ & $6,8\,k\Omega$ & Resistor \\
        $R_3$ & $100\,k\Omega$ & Potmeter \\
        $R_4$ & $10\,k\Omega$ & Potmeter \\
        \bottomrule
    \end{tabular}
    \label{tab:Komponenter_i_posisjonsmaler}
\end{table}

\subsection{Resultater}

\todo[inline]{RIIIIP, vi har jo ikke resultater for dette}

\subsection{Diskusjon}

\todo[inline]{Dette er vrient uten resultater}