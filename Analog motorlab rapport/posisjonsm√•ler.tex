\section{Posisjonsmåler}\label{sec:pos_måler}
Dersom du vil legge inn matematikk, kan du bruke equation-miljøet, som under. Her har vi et uttrykk for en
PI-regulator
\begin{equation}
    u(t) = K_p (x(t)-x_r(t)) + K_i\int_{\tau=t_0}^t (x(\tau)-x_r(\tau))d\tau,
    \label{eq:PI_regulator_OLD}
\end{equation}
hvor $u$ er pådraget, $x$ er tilstanden vi regulerer, $t$ er tiden, $x_r$ er referansen, og $K_p$ og $K_i$ er regulatorkonstanter. Det er en fordel å la likninger være en del av teksten (som over), ettersom dette typisk gjør rapporten mer lettlest. Merk også at vi bruker tegnsetting på samme måte som i vanlige setninger. Altså,
hvis likningen er slutten på en setning, så skal det være et punktum på slutten. Over har jeg brukt komma, ettersom jeg fortsatte setningen etter likningen. Det er også viktig å gjøre rede for hva likningen viser,
og forklare symbolbruken i likningen, som gjort over.

Dersom vi ønsker å skrive matematiske uttrykk (eller bare variabler) i løpende tekst (utenfor likning),
kan vi gjøre det med å skrive \$uttrykk\$.

Referanser til likninger skal gjøres med parantes rundt likningsnummeret. Dette ordnes automatisk dersom dere
bruker kommandoen \texttt{\textbackslash eqref}.
