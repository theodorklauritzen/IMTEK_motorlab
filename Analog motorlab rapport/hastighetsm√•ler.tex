\section{Hastighetsmåler}
\subsection{Teori}
Differensialforsterkeren i \autoref{fig:differensialforsterker} er er operasjonsforsterkerkrets som forsterker opp forskjellen på de to signalene/spenningene som blir satt inn på kretsen. En differensialforsterker av typen i \autoref{fig:differensialforsterker} vil med en ideell operasjonsforsterker kunne utrykkes ved transferfunksjonen\cite{Johnson}:
\begin{equation}
    V_{out} = \frac{R_2}{R_1}(V_2-V_1)
    \label{eq:differensialforsterker}
\end{equation}

En Instrumenteringsforsterker som vist i \autoref{fig:instrumenteringsforsterker} er en mer komplisert versjon av en differensialforsterker, men som har noen viktige fordeler, hver av inngangene i instrumenteringsforsterkeren går rett inn i en operasjonsforsterker slik at inngangsimpedansen holdes høy, dette gjør essensiellt at signalet blir buffret før det sendes inn i differensialforsterkeren bestående av $R_1$, $R_2$ og $OP3$. Forsterkningen i kretsen kan også endres ved å endre på $R_G$. Instrumenteringsforsterkeren i \autoref{fig:instrumenteringsforsterker} kan dersom man antar en ideell operasjonsforsterker uttrykkes ved transferfunksjonen:
\begin{equation}
    V_{out} = (1+\frac{2R_1}{R_G})(\frac{R_3}{R_2})(V_2-V_1)
    \label{eq:instrumenteringsforsterker}
\end{equation}


\begin{figure} [h]
    \centering
    
\begin{circuitikz} [scale=0.8,transform shape]
\ctikzset{resistor = european}

\node[op amp](OP){$OP1$};

\draw (OP.-)
    -- ++(-0.5,0)
    coordinate(N1)
    to[short,*-] ++(-0.5,0)
    to[R,l_=$R_1$] ++(-2,0)
    -- ++(-0.5,0)
    to[short,o-,l=$v_1$] ++(0,0);

\draw (OP.+)
    -- ++(-0.5,0)
    coordinate(N2)
    to[short,*-] ++(-0.5,0)
    to[R=$R_1$] ++(-2,0)
    -- ++(-0.5,0)
    to[short,o-,l=$v_2$] ++(0,0);
    
\draw (N1)
    -- ++(0,1)
    coordinate(temp1)
    to[R=$R_2$] (temp1 -| OP.out)
    -- (OP.out)
    to[short,*-] ++(1,0)
    to[short,o-,l_=$v_{out}$] ++(0,0);

\draw (N2)
    -- ++(0,-0.5)
    to[R=$R_2$] ++(0,-2)
    node[ground]{} ++(0,0);

\end{circuitikz}

    \caption{Differensiell operasjonsforsterer. Figur hentet fra \cite{Johnson}}
    \label{fig:differensialforsterker}
\end{figure}


\begin{figure} [h]
    \centering
    \begin{circuitikz} [scale=0.6, transform shape]
\ctikzset{resistor = european}

\node [op amp, noinv input up] (OP){$OP1$};

% Boi dine internenavn matcher ikke de sylige navnene, det går fint tho

\draw (OP.+)
    -- ++(-2,0)
    to[short,o-,l=$v_1$] ++(0,0);

\draw (OP.-)
    -- ++(-1,0)
    to[short,-*] ++(0, -1)
    coordinate(o1)
    to[R,l_=$R_1$] (o1 -| OP.out)
    -- (OP.out)
    to[short,*-] ++(0.5,0)
    -- ++(0,-1)
    to[R,l=$R_2$] ++(2,0)
    coordinate(x2)
    to[short,*-] (x2)
    -- ++(0,-1)
    -- ++(1,0)
    node[op amp, anchor=-](OP2){$OP3$};

\draw (x2)
    to[R,l=$R_3$] (x2 -| OP2.out)
    -- (OP2.out)
    to[short,*-] ++(1,0)
    to[short,o-,l_=$v_{out}$] ++(0,0);

\draw (o1)
    -- ++(-0.7,0)
    to[potentiometer,l_=$R_G$, n=RG] ++(0,-2)
    -- ++(0.7,0)
    coordinate(o2)
    to[short,*-] (o2)
    -- ++(0,-1)
    -- ++(1,0)
    node[op amp, anchor=-](OP3){$OP2$};

\draw (o1)
    |- (RG.wiper);
    
\draw (OP3.+)
    -- ++(-2,0)
    to[short,o-,l=$v_2$] ++(0,0);

\draw (o2)
    to[R,l=$R_1$] (o2 -| OP3.out)
    -- (OP3.out)
    to[short,*-] ++(0.5,0)
    coordinate(temp2)
    -- (OP2.+ -| temp2)
    to[R,l=$R_2$] ++(2,0)
    coordinate(x3)
    to[short,*-] (OP2.+)
    (x3)
    -- ++(0,-0.5)
    to[R,l=$R_3$] ++(0,-2)
    node[ground] {}++(0,0);
    
    
    
    

\end{circuitikz}
    \caption{Intrumenteringsforsterker. Figur hentet fra \cite{Johnson}}
    \label{fig:instrumenteringsforsterker}
\end{figure}








\subsection{Metode}

For å måle hastigheten, var et tachometer montert på den ene enden av motoren. Dette tachometeret ga ut hastigheten til motoren i form av en differensiell spenning i intervallet $\pm${\SI{1.79}{\volt}} . For å oppnå ønsket spenningsnivå i $\omega_m$ på $\pm${\SI{10}{\volt}} ble en differensialforsterker (\autoref{fig:differensialforsterker}) og en instrumenteringsforsterker (\autoref{fig:instrumenteringsforsterker}) implementert på studentkortet.
Hensikten med å implementere både differensialforsterkeren og instrumenteringsforsterkeren var å sammenlikne de to spesielt med tanke respons til variasjon i utgangsimpedanse fra tachometeret.
To jumpere ble plassert på tachometerets to utganger $V_{11}$ og $V_{12}$ for å raskt kunne bytte mellom de to forsterkerkretsene. I differensialforsterkeren brukes resistorer med vedier for $R_1$ og $R_2$ gitt i \autoref{tab:differensiellforsterker}, og forsterkningen er gitt ved $\frac{R_2}{R_1} = \frac{56k\Omega}{10k\Omega}$(\autoref{eq:differensialforsterker}).

I instrumenteringsforsterkeren brukes motstander med verdiene gitt i \autoref{tab:Instrumenteringsforsterker}. Instumenteringsforsterkeren kan endres forsterkning på ved å variere på motstanden $R_G$, hvor dersom $R_G$ settes til {\SI{0}{\ohm}}, vil utgangsspenningen være avgrenset av operasjonsforsterkerens forsyningsspenning på $\pm${\SI{15}{\volt}} . $R_G$ ble beregnet til ca. {\SI{2.2}{\kilo\ohm}} for å oppnå $\pm${\SI{10}{\volt}} på utgangen, før $R_G$ så ble manuelt stillt for å oppnå best mulig resultat på forsterkningen. Begge forsterkerene ble så testet med både høy og lav utgangsimpedans på tachometeret, hvor målinger ble gjort på $\omega_m$ for å sammenligne responser.

\begin{table}[h]
	\centering
    \caption{Motstander i differensiellforsterkeren}
	\begin{tabular}{lll}
		\toprule
		Størrelse & Verdi & Type \\
		\midrule
        $R_1$ & $10\,k\Omega$& Resistor \\
        $R_2$ & $56\,k\Omega$ & Resistor\\
		\bottomrule
	\end{tabular}
\label{tab:differensiellforsterker}
\end{table}

\begin{table}[h]
	\centering
    \caption{Motstander i Instrumenteringsforsterkeren}
	\begin{tabular}{lll}
		\toprule
		Størrelse & Verdi & Type \\
		\midrule
        $R_1, R_3$ & $10\,k\Omega$& Resistor \\
        $R_2$ & $18\,k\Omega$ & Resistor\\
        $R_G$ & $10\,k\Omega$ & Potmeter\\
		\bottomrule
	\end{tabular}
    \label{tab:Instrumenteringsforsterker}
\end{table}







\subsection{Resultater}

Ved å flippe mellom høy og lav impedans på utgangen til tachometeret ble det med instrumenteringsforsterkeren ikke observert noen tydelig endring i spenningsintervallet. Med differensialforsterkeren og utgangsimpedansen til tachometeret lav, ble det observert et intervall i $\omega_m$ på omtrent $\pm${\SI{10}{\volt}}. Med utgangsimpedansen satt til høy på tachometeret, falt utgangsspenningen $\omega_m$ fra differensialforsterkeren til omtrent $\pm${\SI{100}{\milli\volt}}.






\subsection{Diskusjon}

Med lav utgangsimpedans fra tachometeret ga både differensialforsterkeren og intrumenteringsforsterkeren en $\omega_m$ på intervallet $\pm${\SI{10}{\volt}}, som følger forventningen. Begge forsterkerne viser tendenser til offset på outputen, som kan skyldes av at operasjonsforsterkerene har litt offset i seg, eller at det er litt offset på utgangssignalet til tachometeret. 
Hverken differensial- eller instrumenteringsforsterkerkretsen hadde mulighet for å justere offset, men intervallet var innenfor akseptable marginer.

Med høy utgangsimpedans på tachometeret ga instrumenteringsforsterkeren fortsatt ut $\omega_m$ i intervallet $\pm${\SI{10}{\volt}}, dette skjer på grunn av at signalet fra tachometeret går rett inn i en operasjonsforsterker med veldig høy inngangsimpedans slik at kretsen trekker lite strøm og dermed ikke ble påvirket i noen særlig grad av den høyere impedansen. 
Differensialforsterkeren på sin side ble sterkt påvirket av endringen i utgangsimpedans. Den ga en $\omega_m$ på ca. $\pm${\SI{100}{\milli\volt}} når utgangsimpedansen ble satt høy. Dette følger teorien ettersom du får en spenningsdeling mellom en motstand på {\SI{1}{\mega\ohm}} og $R_1$\autoref{fig:differensialforsterker}.

På grunn av instrumenteringsforsterkerens høyere robusthet, var det denne som ble brukt videre i motorlaben for å ende opp med et best mulig resultat til slutt.
